\documentclass{jsarticle}
\usepackage{amsmath,amssymb}
\usepackage{enumerate}
\usepackage{breqn}
\usepackage{amsthm}
\usepackage{ascmac}
\theoremstyle{definition}
\usepackage{bm}
\newtheorem*{pf}{証明}
%\newtheorem{Ex}{演習問題}
%\newtheorem*{Ex*}{演習問題}
\newtheoremstyle{mystyle}%   % スタイル名
    {}%                      % 上部スペース
    {}%                      % 下部スペース
    {\normalfont}%           % 本文フォント
    {}%                      % インデント量
    {\bf}%                   % 見出しフォント
    {}%                      % 見出し後の句読点, '.'
    { }%                     % 見出し後のスペース, ' ' or \newline
    {\underline{\thmname{#1}\thmnumber{#2}\thmnote{(#3)}}}%
                             % 見出しの書式 (can be left empty, meaning `normal')
\theoremstyle{mystyle} % スタイルの適用
\newtheorem*{Def}{Def}
\newtheorem*{theo}{Theorem}
\newtheorem*{lem}{Lemma}
\newtheorem*{ex}{Example}
\newtheorem*{col}{Corollary}
\renewcommand{\footnotesize}{\normalsize}
\usepackage{latexsym}
\def\qed{\hfill$\Box$}

\begin{document}
\large
\section*{補題の証明}
\begin{itembox}[l]{補題1}
任意の$x,y\in \mathbb{R}^p$に対して
$$\frac{2}{L}f(x)-f(p(y))\geq \|p(y)-y\|_2^2+2(y-x)^T(p(y)-y)$$
が成立する.
\end{itembox}
\begin{proof}
関数$g$の微分可能性より,Taylorの定理から任意の$x,y\in \mathbb{R}^p$に対して
\begin{align}
\label{taylor}
g(x)&=g(y)+(x-y)^T\nabla g(y)+\frac{1}{2}(x-y)^T\nabla^2 g(y+\theta(x-y))(x-y)
\end{align}
を満たす実数$\theta$が存在する.さらに定数$L$の定め方より
\begin{align}
\label{lem11}
(x-y)^T\nabla^2 g(y+\theta(x-y))(x-y)\leq L\|x-y\|_2^2
\end{align}
が成立することがわかる.以上(\ref{taylor}),(\ref{lem11})式より不等式
\begin{align*}
g(x)\leq g(y) +(x-y)^T \nabla g(y)+\frac{L}{2}\|x-y\|_2^2
\end{align*}
が成立するので,両辺に$h(x)$を加えることで
\begin{equation}
\begin{split}
\label{lem15}
f(x)&=g(x)+h(x)\\
&\leq g(y) +(x-y)^T \nabla g(y)+\frac{L}{2}\|x-y\|_2^2+h(x)=Q(x,y)\\
\therefore f(x)&\leq Q(x,y)
\end{split}
\end{equation}
が任意の$x,y\in \mathbb{R}^p$に対して成立することがわかる.
\\

ここで,関数$Q(x,y)$を$x$で劣微分し,$0$を要素として含むという条件は,$\gamma(y)$を$x=p(y)$における$h(x)$の劣微分の要素の1つとして
\begin{align}
\label{lm12}
\nabla g(y)+L (p(y)-y)+\gamma(y)=0
\end{align}
と書くことができ,さらに$g,h$がどちらも凸関数であったことから任意の$x,y\in \mathbb{R}^p$に対して
\begin{align}
\label{lm13}
g(x)&\geq g(y)+(x-y)^T\nabla g(y)\\
\label{lm14}
h(x)&\geq h(p(y))+(x-p(y))^T\gamma(y)
\end{align}
が成立することがわかる.以上(\ref{lm12}),(\ref{lm13}),(\ref{lm14})式を用いることで,不等式

\begin{align*}
f(x)-Q(p(y),y)&=g(x)+h(x)-Q(p(y),y)\\
&\geq g(y)+(x-y)^T\nabla g(y)+h(p(y))+(x-p(y))^T\gamma(y)\\
&\hspace{1.5cm}-\left\{g(y) +(p(y)-y)^T \nabla g(y)+\frac{L}{2}\|p(y)-y\|_2^2+h(p(y))\right\}\\
&=-\frac{L}{2}\|p(y)-y\|_2^2+(x-p(y))^T(\nabla g(y)+\gamma (y))\\
&=-\frac{L}{2}\|p(y)-y\|_2^2+L(x-y+y-p(y))^T(y-p(y))\\
&=\frac{L}{2}\|p(y)-y\|_2^2+L(x-y)^T(y-p(y))\\
\therefore f(x)-Q(p(y),y)&\geq \frac{L}{2}\|p(y)-y\|_2^2+L(x-y)^T(y-p(y))
\end{align*}
が成立することがわかる.さらに不等式(\ref{lem15})を上式に適用すれば
\begin{align*}
f(x)-f(p(y))&\geq \frac{L}{2}\|p(y)-y\|_2^2+L(x-y)^T(y-p(y))\\
&=\frac{L}{2}\|p(y)-y\|_2^2+L(y-x)^T(p(y)-y)
\end{align*}
となることがわかり,これが求めたかった不等式である.
\end{proof}

\begin{itembox}[l]{補題2}
FISTAで考える列$\{\alpha_k\}$と得られる列$\{\beta_k\}$,及び最適解$\beta_{\ast}$に対して次式が成立する
\begin{align*}
\frac{2}{L}\left\{\alpha_k^2(f(\beta_k)-f(\beta_{\ast}))-\alpha_{k+1}^2(f(\beta_{k+1})-f(\beta_{\ast}))\right\}&\geq \|t_{k+1}\|_2^2-\|t_k\|_2^2
\end{align*}
ただし$t_k:=\alpha_k\beta_k-(\alpha_k-1)\beta_{k-1}-\beta_{\ast}$とする.
\end{itembox}
\begin{proof}
補題1において$x=\beta_k,y=\gamma_{k+1}$とおくことで
\begin{align}
\label{lem21}
\frac{2}{L}(f(\beta_k)-f(\beta_{k+1}))\geq \|\beta_{k+1}-\gamma_{k+1}\|_2^2+2(\gamma_{k+1}-\beta_k)^T(\beta_{k+1}-\gamma_{k+1})
\end{align}
が成立する.同様に補題1において$x=\beta_{\ast},y=\gamma_{k+1}$とおくことで
\begin{align}
\label{lem22}
\frac{2}{L}(f(\beta_{\ast})-f(\beta_{k+1}))\geq \|\beta_{k+1}-\gamma_{k+1}\|_2^2+2(\gamma_{k+1}-\beta_{\ast})^T(\beta_{k+1}-\gamma_{k+1})
\end{align}
が成立する.(\ref{lem21})式の両辺を$\alpha_{k+1}-1$倍し,(\ref{lem22})式の両辺を加えることを考えると,(\ref{lem21})式の両辺の$\alpha_{k+1}-1$倍は

\begin{align*}
&\frac{2}{L}(\alpha_{k+1}-1)(f(\beta_k)-f(\beta_{k+1}))\geq (\alpha_{k+1}-1)\|\beta_{k+1}-\gamma_{k+1}\|_2^2+2(\alpha_{k+1}-1)(\gamma_{k+1}-\beta_k)^T(\beta_{k+1}-\gamma_{k+1})\\
\Leftrightarrow &\frac{2}{L}\left\{(\alpha_{k+1}-1)f(\beta_k)-\alpha_{k+1}f(\beta_{k+1})+f(\beta_{k+1})\right\}\geq \alpha_{k+1}\|\beta_{k+1}-\gamma_{k+1}\|_2^2-\|\beta_{k+1}-\gamma_{k+1}\|_2^2\\
&\hspace{9cm}+2(\gamma_{k+1}-\beta_k)^T(\beta_{k+1}-\gamma_{k+1})
\end{align*}
であるため,上式に(\ref{lem22})式を加えることで
\begin{align*}
\frac{2}{L}&\{(\alpha_{k+1}-1)f(\beta_k)-\alpha_{k+1}f(\beta_{k+1})+f(\beta_{\ast})\}\\
&\geq \alpha_{k+1}\|\beta_{k+1}-\gamma_{k+1}\|_2^2
+2(\alpha_{k+1}-1)(\gamma_{k+1}-\beta_k)^T(\beta_{k+1}-\gamma_{k+1})+2(\gamma_{k+1}-\beta_{\ast})^T(\beta_{k+1}-\gamma_{k+1})\\
&=\alpha_{k+1}\|\beta_{k+1}-\gamma_{k+1}\|_2^2+2\{(\alpha_{k+1}-1)(\gamma_{k+1}-\beta_k)^T+(\gamma_{k+1}-\beta_{\ast})^T\}(\beta_{k+1}-\gamma_{k+1})\\
&=\alpha_{k+1}\|\beta_{k+1}-\gamma_{k+1}\|_2^2+2\langle \beta_{k+1}-\gamma_{k+1},\alpha_{k+1}\gamma_{k+1}-(\alpha_{k+1}-1)\beta_k-\beta_{\ast} \rangle
\end{align*}
\begin{equation}
\begin{split}
\label{lem23}
\therefore \frac{2}{L}&\{(\alpha_{k+1}-1)f(\beta_k)-\alpha_{k+1}f(\beta_{k+1})+f(\beta_{\ast})\}\\
&\hspace{3cm}\geq \alpha_{k+1}\|\beta_{k+1}-\gamma_{k+1}\|_2^2+2\langle \beta_{k+1}-\gamma_{k+1},\alpha_{k+1}\gamma_{k+1}-(\alpha_{k+1}-1)\beta_k-\beta_{\ast} \rangle
\end{split}
\end{equation}
が成立し,上式の両辺を$\alpha_{k+1}$倍することで
\begin{equation}
\begin{split}
\label{lem24}
\frac{2}{L}&\{(\alpha_{k+1}^2-\alpha_{k+1})f(\beta_k)-\alpha_{k+1}^2f(\beta_{k+1})+f(\beta_{\ast})\}\\
&\hspace{3cm}\geq \|\alpha_{k+1}(\beta_{k+1}-\gamma_{k+1})\|_2^2+2\alpha_{k+1}\langle \beta_{k+1}-\gamma_{k+1},\alpha_{k+1}\gamma_{k+1}-(\alpha_{k+1}-1)\beta_k-\beta_{\ast} \rangle
\end{split}
\end{equation}
が得られる.ここで,$\alpha_{k}$の定め方より
\begin{align*}
\alpha_{k+1}&=\frac{1+\sqrt{1+4\alpha_k^2}}{2}\\
2\alpha_{k+1}-1&=\sqrt{1+4\alpha_k^2}\\
4\alpha_{k+1}^2-4\alpha_{k+1}+1&=1+4\alpha_k^2\\
\therefore \alpha_{k+1}^2-\alpha_{k+1}&=\alpha_k^2
\end{align*}
となるので,これを(\ref{lem24})式に適用することで
\begin{equation}
\begin{split}
\label{lem25}
\frac{2}{L}&\{\alpha_k^2f(\beta_k)-\alpha_{k+1}^2f(\beta_{k+1})+f(\beta_{\ast})\}\\
&\hspace{3cm}\geq \|\alpha_{k+1}(\beta_{k+1}-\gamma_{k+1})\|_2^2+2\alpha_{k+1}\langle \beta_{k+1}-\gamma_{k+1},\alpha_{k+1}\gamma_{k+1}-(\alpha_{k+1}-1)\beta_k-\beta_{\ast} \rangle
\end{split}
\end{equation}
が成立することがわかる.一般に,$a,b,c\in \mathbb{R}^p$に対して
\begin{align}
\label{lem26}
\|b-a\|_2^2+2\langle b-a,a-c \rangle&=\|b-c\|_2^2-\|a-c\|_2^2
\end{align}
が成立することから,$a=\alpha_{k+1}\gamma_{k+1},b=\alpha_{k+1}\beta_{k+1},c=(\alpha_{k+1}-1)\beta_k+\beta_{\ast}$として(\ref{lem25})式に(\ref{lem26})式を適用すれば
\begin{align*}
\frac{2}{L}&\{\alpha_k^2f(\beta_k)-\alpha_{k+1}^2f(\beta_{k+1})+f(\beta_{\ast})\}\\
&\geq \|b-a\|_2^2+2\langle  b-a,a-c  \rangle\\
&=\|b-c\|_2^2-\|a-c\|_2^2\\
&=\|\alpha_{k+1}\beta_{k+1}-(\alpha_{k+1}-1)\beta_k-\beta_{\ast}\|_2^2-\|\alpha_{k+1}\gamma_{k+1}-(\alpha_{k+1}-1)\beta_k-\beta_{\ast}\|_2^2\\
&=\|\alpha_{k+1}\beta_{k+1}-(\alpha_{k+1}-1)\beta_k-\beta_{\ast}\|_2^2-\|\alpha_k \beta_k+(\alpha_k-1)\beta_{k-1}-\beta_{\ast}\|_2^2\\
&=\|t_{k+1}\|_2^2-\|t_k\|_2^2
\end{align*}
が成立することがわかる.ただし,4番目の式変形は(3.20)式より
\begin{equation}
\begin{split}
\label{lem27}
\gamma_{k+1}&=\beta_k+\frac{\alpha_k-1}{\alpha_{k+1}}(\beta_k-\beta_{k-1})\\
\alpha_{k+1}\gamma_{k+1}&=\alpha_{k+1}\beta_k+(\alpha_k-1)(\beta_k-\beta_{k-1})
\end{split}
\end{equation}
が成立することを用いた.(\ref{lem27})式の最左辺と最右辺を見ることで,不等式
\begin{align*}
\frac{2}{L}&\{\alpha_k^2f(\beta_k)-\alpha_{k+1}^2f(\beta_{k+1})+f(\beta_{\ast})\}\geq \|t_{k+1}\|_2^2-\|t_k\|_2^2
\end{align*}
が得られる.これは求める不等式であった.
\end{proof}
\begin{itembox}[l]{補題3}
ある実数$c$が存在して,2つの非負値実数列$\{a_k,b_k\}$が任意の自然数$k\geq 1$について
\begin{align*}
a_k-a_{k+1}\geq b_{k+1}-b_k\;かつ\; a_1+b_1\leq c
\end{align*}
を満たすならば,任意の自然数$k$に対して$a_k\leq c$が成立する.
\end{itembox}
\begin{proof}
条件式を変形すると
$$a_k+b_k\geq a_{k+1}+b_{k+1}$$
となり,これは実数列$\{a_k+b_k\}$が単調減少列であることを示している.これと$c\geq a_1+b_1$であることより,任意の自然数$k$に対して
\begin{align*}
c\geq a_1+b_1\geq \cdots \geq a_k+b_k
\end{align*}
が成立する.$\{a_k\},\{b_k\}$の非負値性より,任意の自然数$k$に対して$a_k\leq c$がいえる
\end{proof}

\begin{itembox}[l]{補題4}
FISTAで考える列$\{\alpha_k\}$は任意の自然数$k\geq 1$に対して
$$\alpha_k\geq \frac{k+1}{2}$$
が成立する.
\end{itembox}
\begin{proof}
帰納法で証明する;(3.19)式より任意の自然数$k\geq1$に対して
\begin{align*}
\alpha_{k+1}&=\frac{1+\sqrt{1+4\alpha_{k}^2}}{2}\\
&\geq \frac{1+\sqrt{1+(k+1)^2}}{2}\quad(\because 帰納法の仮定)\\
&\geq \frac{1+\sqrt{(k+1)^2}}{2}=\frac{(k+1)+1}{2}
\end{align*}
となるので
$$\alpha_{k+1}\geq \frac{(k+1)+1}{2}$$
が成立する.また$k=1$のときは
$$\alpha_1=1=\frac{1+1}{2}$$
となるので成立している.以上より任意の自然数$k\geq 1$に対して
$$\alpha_k\geq \frac{k+1}{2}$$
が成立することが言える.
\end{proof}



\end{document}