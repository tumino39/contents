\documentclass{jsarticle}
\usepackage{amsmath,amssymb}
\usepackage{enumerate}
\usepackage{breqn}
\usepackage{amsthm}
\theoremstyle{definition}
%\newtheorem{Ex}{演習問題}
%\newtheorem*{Ex*}{演習問題}
\usepackage{latexsym}
\def\qed{\hfill$\Box$}
\pagestyle{myheadings}

\title{レポート}
\date{}
\author{201910790 \quad 海野 哲也}
\begin{document}
\markright{201910790 海野 哲也}

\large
\setcounter{section}{3}
\setcounter{subsection}{4}
\subsection{オーバーラップグループLasso}
説明変数に関して$\{1,2,3\},\{3,4,5\}$のように,グループに含まれる変数が重複する場合にもグループLassoの処理を構成したい.グループ$k=1,\cdots,K$において,変数に対する係数$\beta_1,\cdots,\beta_p$のうち,用いない変数$\beta_k$については$0$とおく.また$\mathbb{R}^p\ni\beta=\displaystyle\sum_{k=1}^K\theta_k$を満たす$\theta_1,\cdots,\theta_k\in \mathbb{R}^p$を用意する.そして,与えられる$X\in\mathbb{R}^{N\times p},y\in\mathbb{R}^N$から
$$L(\theta)=\frac{1}{2}\left\|y-X\sum_{k=1}^K\theta_k\right\|_2^2+\lambda\sum_{k=1}^K\|\theta_k\|_2$$
を最小にする$\theta$を求めることを考える.\\

例として,$p=5,K=2$で$\beta_3$のみがオーバーラップしている場合を考える.この場合,$\theta_1,\theta_2$は以下のようになる
$$\theta_1=\left[\begin{array}{c}
\beta_1\\
\beta_2\\
\beta_{3,1}\\
0\\
0\end{array}\right],\quad \theta_2=\left[\begin{array}{c}
0\\
0\\
\beta_{3,2}\\
\beta_4\\
\beta_5\end{array}\right]\quad(\beta_3 = \beta_{3,1}+\beta_{3,2}).$$

ここで,$X\in\mathbb{R}^{N\times 5}$の最初の3列を$X_1\in \mathbb{R}^{N\times 3}$,最後の3列を$X_2\in \mathbb{R}^{N\times 3}$と書き,$L$について$\theta_1,\theta_2$の最初の3成分$\gamma_1,\gamma_2$で劣微分をとると
\begin{align*}
\frac{\partial L}{\partial \gamma_1}&=\displaystyle\left[\begin{array}{c}
\frac{\partial L}{\partial \beta_1}\\
\frac{\partial L}{\partial \beta_2}\\
\frac{\partial L}{\partial \beta_{3,1}}
\end{array}\right]=-X_1^T(y - X_1 \gamma_1)+\lambda\partial \|\gamma_1\|_2\\
\frac{\partial L}{\partial \gamma_2}&=\left[\begin{array}{c}
\frac{\partial L}{\partial \beta_{3,2}}\\
\frac{\partial L}{\partial \beta_4}\\
\frac{\partial L}{\partial \beta_{5}}
\end{array}\right]=-X_2^T(y - X_2 \gamma_2)+\lambda\partial \|\gamma_2\|_2
\end{align*}
となることがわかるので,グループ数が1の場合での全体の微分が$0$を含むとした(3.8)式は
\begin{align*}
-X_1^T(y - X \theta_1)+\lambda\partial \|\theta_1\|_2&=0\\
-X_2^T(y - X \theta_2)+\lambda\partial \|\theta_2\|_2&=0
\end{align*}
に相当することがわかる.したがって$\theta_j=0(j=1,2)$で最小となるとき$\|X_j^Ty\|_2\leq \lambda$が成立することから,結局オーバーラップがあった場合でも通常のグループLassoと同じ手法で最適化ができることがわかる.\\


\subsection{目的変数が複数個ある場合のグループLasso}
観測データ$X\in \mathbb{R}^{N\times p},\beta\in \mathbb{R}^{p\times K},y\in \mathbb{R}^{N\times K}$から
\begin{align}
\label{l0}
L_0(\beta):=\frac{1}{2}\sum_{i=1}^N\sum_{k=1}^K\left(y_{i,k}-\sum_{j=1}^px_{i,j}\beta_{j,k}\right)^2
\end{align}
として
\begin{align*}
L(\beta):=L_0(\beta)+\lambda\sum_{j=1}^p\|\beta_j\|_2
\end{align*}
を最小にする$\beta\in\mathbb{R}^{p\times K}$を求めたい.上式で$K=1$とすれば,これは第1章で扱ったLassoに相当する.$y_i=[y_{i,1},\cdots,y_{i,K}]$というように目的変数を$K$個に拡張した場合,各$j$について$\beta_j=[\beta_{j,1},\cdots,\beta_{j,K}]$の$K$個のアクティブ・非アクティブのタイミングが同じであることを仮定している.\\

$L(\beta)$の最小化について考える.(\ref{l0})式を$\beta_{j,k}$で偏微分すると
\begin{align*}
\frac{\partial L_0}{\partial \beta_{j,k}}&=\sum_{i=1}^N\{-x_{i,j}(r_{i,k}^{(j)}-x_{i,j}\beta_{j,k})\}\\
&(ただし\;r_{i,k}^{(j)}:=y_{i,k}-\sum_{h\neq j}x_{i,h}\beta_{h,k})
\end{align*}
となることから$L(\beta)$の$\beta_j$による劣微分は
\begin{align}
\label{retsu}
\beta_j\sum_{i=1}^Nx_{i,j}^2-\sum_{i=1}^Nx_{i,j}r_i^{(j)}+\lambda\partial\|\beta_j\|_2
\end{align}
となることがわかる.もし$\hat{\beta_j}=0$が最適解ならば,(\ref{retsu})が$0$となるのは
\begin{align*}
\lambda\partial\|\beta_j\|_2&\ni\sum_{i=1}^Nx_{i,j}r_i^{(j)}\\
\therefore \left\|\sum_{i=1}^Nx_{i,j}r_i^{(j)}\right\|_2&\leq\lambda
\end{align*}
のときであり,$\hat{\beta}_j\neq 0$が最適解であるならば,グループ数が1の場合のグループLassoと同様の論法により
\begin{align*}
\hat{\beta}_j=\frac{1}{\sum_{i=1}^Nx_{i,j}^2}\left(1-\frac{\lambda}{\|\sum_{i=1}^Nx_{i,j}r_i^{(j)}\|_2}\right)\sum_{i=1}^Nx_{i,j}r_i^{(j)}
\end{align*}
となることがわかる.以上より(\ref{retsu})式を0にする$\hat{\beta}_j$は
\begin{align*}
\hat{\beta}_j=\frac{1}{\sum_{i=1}^Nx_{i,j}^2}\left(1-\frac{\lambda}{\|\sum_{i=1}^Nx_{i,j}r_i^{(j)}\|_2}\right)_+\sum_{i=1}^Nx_{i,j}r_i^{(j)}
\end{align*}
とかけることがわかる.



\end{document}