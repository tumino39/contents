\documentclass{jsarticle}
\usepackage{amsmath,amssymb}
\usepackage{enumerate}
\usepackage{breqn}
\usepackage{amsthm}
\theoremstyle{definition}
\usepackage{bm}
%\newtheorem{Ex}{演習問題}
%\newtheorem*{Ex*}{演習問題}
\newtheoremstyle{mystyle}%   % スタイル名
    {}%                      % 上部スペース
    {}%                      % 下部スペース
    {\normalfont}%           % 本文フォント
    {}%                      % インデント量
    {\bf}%                   % 見出しフォント
    {}%                      % 見出し後の句読点, '.'
    { }%                     % 見出し後のスペース, ' ' or \newline
    {\underline{\thmname{#1}\thmnumber{#2}\thmnote{(#3)}}}%
                             % 見出しの書式 (can be left empty, meaning `normal')
\theoremstyle{mystyle} % スタイルの適用
\newtheorem*{Def}{Def}
\newtheorem*{theo}{Theorem}
\newtheorem*{lem}{Lemma}
\newtheorem*{ex}{Example}
\newtheorem*{col}{Corollary}
\renewcommand{\footnotesize}{\normalsize}
\usepackage{latexsym}
\def\qed{\hfill$\Box$}

\begin{document}
\large
2次元の場合について考える.今,$X$は
\begin{align*}
X=\begin{pmatrix}
x_{1,1} & x_{1,2}\\
\vdots & \vdots \\
x_{N,1} & x_{N,2}
\end{pmatrix}
\end{align*}
であるとする.また,$X$は中心化されていると仮定する.すなわち
$$\bar{x_k}:=\frac{1}{N}\sum_{i=1}^Nx_{i,k}=0\quad(k=1,2)$$
であるとする.\\

$X^TX$の2次形式を最大にする単位ベクトル$u\in\mathbb{R}^2$を考えたい.これはラグランジュの未定乗数法より,ある実数$\lambda'$が存在して
\begin{align*}
F(u,\lambda):=u^TX^TXu-\lambda(1-\|u\|_2^2)
\end{align*}
に対して
$$\frac{\partial F}{\partial u}(u,\lambda')=\frac{\partial F}{\partial \lambda}(u,\lambda')=0$$
を満たしている.ここで,$\displaystyle \frac{\partial F}{\partial u}(u,\lambda)=0$であることより
\begin{align*}
2X^TXu-2\lambda 'u&=0\\
\therefore X^TXu=\lambda' u
\end{align*}
が成立している.したがって求める$u$は存在すれば$X^TX$の固有値となることがわかる.また,上式の両辺に$u^T$をかけることで
$$u^TX^XXu=u^T \lambda u=\lambda\|u\|_2^2=\lambda$$
となることがわかる.上式より$X^TX$の2次形式の最大値は$X^TX$の最大固有値であることがわかる.\\

また
\begin{align*}
u^T X^TXu=\sum_{i=1}^N(x_{i,1}u_1+x_{i,2}u_2)^2
\end{align*}
であることより,上式の左辺は$X$の各行ベクトルを$u$に射影したものの分散に一致することがわかる.したがって$X^TX$の最大固有値は,$Xu$の分散が最大になるように$u$をとったときの2次形式$u^TX^TXu$に等しいことが言える.\\

図でのイメージは以下のようになる


\end{document}