\documentclass{jsarticle}
\usepackage{amsmath,amssymb}
\usepackage{enumerate}
\usepackage{breqn}
\usepackage{amsthm}
\usepackage{ascmac}
\theoremstyle{definition}
\usepackage{bm}
%\newtheorem{Ex}{演習問題}
\newtheorem*{pf}{証明}
\newtheoremstyle{mystyle}%   % スタイル名
    {}%                      % 上部スペース
    {}%                      % 下部スペース
    {\normalfont}%           % 本文フォント
    {}%                      % インデント量
    {\bf}%                   % 見出しフォント
    {}%                      % 見出し後の句読点, '.'
    { }%                     % 見出し後のスペース, ' ' or \newline
    {\underline{\thmname{#1}\thmnumber{#2}\thmnote{(#3)}}}%
                             % 見出しの書式 (can be left empty, meaning `normal')
\theoremstyle{mystyle} % スタイルの適用
\newtheorem*{Def}{Def}
\newtheorem*{theo}{Theorem}
\newtheorem*{lem}{Lemma}
\newtheorem*{ex}{Example}
\newtheorem*{col}{Corollary}
\renewcommand{\footnotesize}{\normalsize}
\usepackage{latexsym}
\def\qed{\hfill$\Box$}

\begin{document}
\large
\section*{近接勾配法}
(3.5)式
\begin{align}
\tag{3.5}
\frac{1}{2}\|y-X\beta\|_2^2+\lambda\|\beta\|_2
\end{align}
の最小化問題に対するアプローチとして,(3.10)式,(3.11)式
\begin{align}
\tag{3.10}
\gamma&:=\beta +\nu X^T(y-X\beta)\\
\tag{3.11}
\beta&=\left(1-\frac{\nu\lambda}{\|\gamma\|_2}\right)_+\gamma
\end{align}
を繰り返すことを前節で考えたが,一般には上式が(3.5)式の最適解に収束することは保証されない。以下では$\nu>0$の値を上手く設定することによって(3.10),(3.11)式のループによって正しい解に収束することを示す.\\

まず$\lambda =0$の場合であれば(3.11)式において
$$\beta=\gamma$$
となるため,(3.10)式と(3.11)式のループは
\begin{align*}
\beta_{t+1}\leftarrow \beta_t+\nu X^T(y-X\beta_t)\gamma
\end{align*}
とかけることに注意したい.ここで,$g(\beta)=\|y-X\beta\|_2^2$と定めれば$\nabla g(\beta)=-X^T(y-X\beta)$とかけることより,上式は
$$\beta_{t+1}\leftarrow \beta_t-\nu \nabla g(\beta_t)$$
と書くことにする.この方法は$g$の減少の大きい方向に$\beta$を更新していくという意味で,勾配法と呼ばれている.\\

以下では,(3.5)式の第2項を$h(\beta):=\lambda \|\beta\|_2$とおいて,$\lambda>0$での勾配法を
\begin{align}
\label{koubai}
\beta_{t+1}\leftarrow \beta_t -\nu \{\nabla g(\beta_t)+\partial h(\beta)\}
\end{align}
とした場合の収束性について考えたい.ここで,関数$h$に対して写像${\rm prox}_h:\mathbb{R}^p\to \mathcal{P}(\mathbb{R}^p)$を
\begin{align*}
{\rm prox}_h(z):={\rm argmin }_{\theta \in \mathbb{R}^p}\left\{\frac{1}{2}\|z-\theta\|_2^2+h(\theta)\right\}
\end{align*}
とすることで,(\ref{koubai})式は
$$\beta_{t+1}\leftarrow {\rm prox}_{\nu h}(\beta_t -\nu \nabla g(\beta_t))$$
とかけること,すなわち(\ref{koubai})式で更新される$\beta$は
$$\varphi(\theta)=\frac{1}{2}\|\beta_t-\nu\nabla g(\beta_t)-\theta\|_2^2+\nu h(\beta)$$
を最小にする$\theta$と一致することを以下で確認する.\\


最初に$\varphi(\theta)$を最小にする$\theta'$について考える;
まず$\beta_t-\nu \nabla g(\beta_t)=0$であるとき,$\varphi(\theta)=\frac{1}{2}\|\theta\|_2^2+\nu h(\theta)\geq 0$であり,等号成立は$\theta=0$のときに限ることより,求める$\theta'$は$\theta'=0$であることがわかる.\\


次に$\beta_t-\nu \nabla g(\beta_t)\neq 0$であるとき,$\theta\neq 0$における$\displaystyle\frac{\partial \varphi(\theta)}{\partial \theta}$を求めると
\begin{align*}
\frac{\partial \varphi(\theta)}{\partial \theta}&=-(\beta_t-\nu\nabla g(\beta_t)-\theta)+\frac{\nu \lambda \theta}{\|\theta\|_2}
\end{align*}
となるので,$\varphi(\theta)$を最小にする$\theta'$は
\begin{align}
\label{kankei}
\beta_t-\nu\nabla g(\beta_t)-\theta'&=\frac{\nu \lambda \theta'}{\|\theta'\|_2}
\end{align}
を満たしている.ここで$\tilde{\beta_t}:=\beta_t-\nu \nabla g(\beta_t)$とおくと,上式はある実数$\alpha\neq 0$が存在して$\theta'$が
$$\theta' = \alpha \tilde{\beta_t}$$
とかけることを意味している.したがってこれを(\ref{kankei})式に代入することで
\begin{align*}
\tilde{\beta_t}-\alpha\tilde{\beta_t}&=\frac{\nu \lambda \alpha}{|\alpha|\|\tilde{\beta_t}\|_2}\tilde{\beta_t}\\
\alpha \tilde{\beta_t}&=\left(1-\frac{\nu \lambda \alpha}{|\alpha|\|\tilde{\beta_t}\|_2}\right)\tilde{\beta_t}
\end{align*}
が成立することがわかり,$\tilde{\beta_t}\neq 0$であることから上式の係数を比較して
\begin{align*}
\alpha&=1-\frac{\nu \lambda \alpha}{|\alpha|\|\tilde{\beta_t}\|_2}\\
&= 1-\frac{\nu\lambda}{\|\tilde{\beta_t}\|}\cdot{\rm sign}(\alpha)
\end{align*}
となることがわかる.ここで,仮に$\alpha<0$であるとすると
\begin{align*}
0>\alpha=1+\frac{\nu\lambda}{\|\tilde{\beta_t}\|}>0
\end{align*}
となってしまい,矛盾が生じる.したがって$\alpha>0$であることがわかるので
$$\alpha=1-\frac{\nu\lambda}{\|\tilde{\beta_t}\|}$$
となることがわかる.これを$\theta'=\alpha\tilde{\beta_t}$に代入することで求める$\theta'$は
\begin{align*}
\theta'&=\left(1-\frac{\nu\lambda}{\|\tilde{\beta_t}\|}\right)\tilde\beta_t\\
&=\tilde\beta_t-\frac{\nu\lambda}{\|\tilde{\beta_t}\|}\tilde\beta_t\\
&=\beta_t-\nu\nabla g(\beta_t)-\frac{\nu\lambda \tilde\beta_t}{\|\tilde\beta_t\|}\\
&=\beta_t-\nu\nabla g(\beta_t)-\nu \frac{\partial h}{\partial \theta}(\tilde \beta_t)\\
\therefore \theta' &= \beta_t-\nu\nabla g(\beta_t)-\nu \frac{\partial h}{\partial \theta}(\tilde \beta_t)
\end{align*}
となることがわかる.\\
以上より$\varphi(\theta)$を最小にする$\theta'$は
\begin{align*}
\theta'=\begin{cases}
\beta_t-\nu\left\{\nabla (\beta_t)- \displaystyle\frac{\partial h}{\partial \theta}(\tilde \beta_t)\right\}&(\beta_t-\nu\nabla (\beta_t)\neq0のとき)\\
0 &(\beta_t-\nu\nabla (\beta_t)=0のとき)
\end{cases}
\end{align*}
と書くことができるが,関数$h$の劣微分が$0$を含むことから改めて
\begin{align*}
\theta' \in \beta_t-\nu\left\{\nabla (\beta_t)- \displaystyle\partial h(\tilde{\beta_t})\right\}
\end{align*}
とかくことができ,これは(\ref{koubai})式と一致することがわかる.\\


上記より,新たに提案された勾配法は
$$\beta_{t+1}\leftarrow {\rm prox}_{\nu h}(\beta_t-\nu\nabla g(\beta_t))$$
とかけることが確認できた.ここで$ {\rm prox}_{\nu h}(\beta_t-\nu\nabla g(\beta_t))$が実際には
$${\rm prox}_{\nu h}(\beta_t-\nu\nabla g(\beta_t))  = \left(1-\frac{\nu\lambda}{\|\tilde\beta_t\|}\right)_+\tilde\beta_t$$とかけていたことから更新式は
\begin{align*}
\gamma &\leftarrow \beta-\nu\nabla g(\beta_t)=\beta +\nu X^T(y-X\beta)\\
\beta&\leftarrow \left(1-\frac{\nu\lambda}{\|\gamma\|_2}\right)_+\gamma
\end{align*}
とかけることがわかる.これは(3.10),(3.11)の更新式に他ならない.\\

最後に$\nu$を上手くとることができれば,上記の更新式によって$f(\beta)$を最小にする$\beta_{\ast}$を求められることを示す.	\\
まず$X^TX$の最大固有値を$L$とすれば,任意の$x,y,z\in \mathbb{R}^p$に対して
$$(x-y)^T\nabla^2g(z)(x-y)\leq L\|x-y\|_2^2$$
が成立することを示す;$X^TX$が対称行列であることから,ある直交行列$P$が存在して
$$P^T(X^TX)P = \begin{pmatrix}
\lambda_1 & &&\\
 & \lambda_2&&\\
 &  & \ddots &\\
 &  & & \lambda_p
\end{pmatrix}$$
が成立する.さらに$L$の定め方より,任意の$x=(x_1,\cdots ,x_p)^T\in \mathbb{R}^p$に対して$I$を$p$次単位行列として
\begin{align*}
x^TP^TLIPx&=L\cdot x^TP^TPx\\
&=L\sum_{i=1}^px_i^2\geq \sum_{i=1}^p \lambda_i x_i^2\\
&=x^T\begin{pmatrix}
\lambda_1 & &&\\
 & \lambda_2&&\\
 &  & \ddots &\\
 &  & & \lambda_p
\end{pmatrix}x\\
&=x^TP^T(X^TX)Px
\end{align*}
が任意の$x \in \mathbb{R}^p$に対して成立していることより,$x\rightarrow P^T(x-y)$とおけば
\begin{align*}
(x-y)^T\nabla^2 g(z)(x-y)&=\left\{P^T(x-y)\right\}^TP^T(X^TX)P\left\{P^T(x-y)\right\}\\
&\leq \left\{P^T(x-y)\right\}^TP^TLIP\left\{P^T(x-y)\right\}^T\\
&=L\|x-y\|_2^2\\
\therefore (x-y)^T\nabla^2 g(z)(x-y)&\leq L\|x-y\|_2^2
\end{align*}
となり,求めたい不等式が導ける.\\

また,(3.13)式について
\begin{align*}
{\rm prox}_{\nu h}(\beta_t - \nu \nabla g(\beta_t))&= {\rm argmin}_{\theta\in \mathbb{R}^p}\left\{\frac{1}{2}\|\beta_t-\nu \nabla g(\beta_t)-\theta\|_2^2+\nu h(\theta)\right\}\\
&={\rm argmin}_{\theta\in \mathbb{R}^p}\left\{\frac{1}{2\nu}\|\beta_t-\nu \nabla g(\beta_t)-\theta\|_2^2+ h(\theta)+g(\beta_t)\right\}\\
&={\rm argmin}_{\theta\in \mathbb{R}^p}\left\{\frac{1}{2\nu}\|\beta_t-\theta\|_2^2+ (\beta_t-\theta)^T\nabla g(\beta_t)+h(\theta)+g(\beta_t)\right\}
\end{align*}
が成立することより,$\nu =\displaystyle\frac{1}{L}$とした(3.13)式を
\begin{align*}
Q(x,y)&:=g(y)+(x-y)^T\nabla g(y)+\frac{L}{2}\|x-y\|_2^2 +h(x)\\
p(y)&:={\rm argmin}_{x\in \mathbb{R}^p}Q(x,y)
\end{align*}
を用いて
\begin{equation}
\label{ISTA}
\beta_{t+1}\leftarrow p(\beta_t)
\end{equation}
と書きなおしておく.\footnote{この方法はISTA(Iterative Shrinkage-Thresholding Algorithm)と呼ばれる}\\

以上で定めた定数$L$,写像$Q,p$に対し,次の命題7が成立する
\begin{itembox}[l]{命題7}
(\ref{ISTA})式によって生成された列$\{\beta_t\}$は,$f$の最小化問題に対する最適解を$\beta_{\ast}$として次式を満たす
$$f(\beta_k)-f(\beta_{\ast})\leq \frac{L\|\beta_0-\beta_{\ast}\|_2^2}{2k}$$
\end{itembox}
証明には以下の補題1を用いる.
\begin{itembox}[l]{補題1}
任意の$x,y\in \mathbb{R}^p$に対して
$$\frac{2}{L}f(x)-f(p(y))\geq \|p(y)-y\|_2^2+2(y-x)^T(p(y)-y)$$
が成立する.
\end{itembox}

\begin{proof}
自然数$k$を任意にとって固定する.補題1で$x=\beta_{\ast},y=\beta_t$とおき,さらに$p(\beta)=\beta_{t+1}$であることを用いれば
\begin{align*}
\frac{2}{L}\left\{f(\beta_{\ast})-f(\beta_{t+1})\right\}&\geq \|\beta_{t+1}-\beta_t\|_2^2+2(\beta_t-\beta_{\ast})^T(\beta_{t+1}-\beta_t)\\
&=\langle \beta_{t+1}-\beta_t,\beta_{t+1}-\beta_t+2(\beta_t-\beta_{\ast})  \rangle\\
&=\langle (\beta_{t+1}-\beta_{\ast})-(\beta_t-\beta_{\ast}),(\beta_{t+1}-\beta_{\ast})+(\beta_t-\beta_{\ast}) \rangle\\
&=\|\beta_{t+1}-\beta_{\ast}\|_2^2-\|\beta_t-\beta_{\ast}\|_2^2
\end{align*}
が任意の自然数$t\leq k$に対して成立することがわかる.したがって上の不等式を$t=0,\cdots,k-1$について足し合わせれば
\begin{equation}
\label{is1}
\frac{2}{L}\left\{kf(\beta_{\ast})-\sum_{t=0}^{k-1}f(\beta_{t+1})\right\}\geq \|\beta_{\ast}-\beta_k\|_2^2-\|\beta_{\ast}-\beta_0\|_2^2
\end{equation}
が成立することがわかる.さらに$x=y=\beta_t$として再び補題1を適用すると
\begin{align*}
\frac{2}{L}\left\{f(\beta_{t})-f(\beta_{t+1})\right\}\geq \|\beta_{t+1}-\beta_t\|_2^2
\end{align*}
となることより,上式の両辺を$t$倍し,さらに$t=0,\cdots ,k-1$について加えることで
\begin{align}
\frac{2}{L}\sum_{t=0}^{k-1}t\left\{f(\beta_t)-f(\beta_{t+1})\right\}&=\frac{2}{L}\sum_{t=0}^{k-1}\left\{tf(\beta_t)-(t+1)f(\beta_{t+1})+f(\beta_{t+1})\right\}\\
\label{is2}
&=\frac{2}{L}\left\{-kf(\beta_k)+\sum_{t=0}^{k-1}f(\beta_{t+1})\right\}\geq \sum_{t=0}^{k-1}t\|\beta_t-\beta_{t+1}\|_2^2
\end{align}
が成立することがわかる.以上で得られた不等式(\ref{is1}),(\ref{is2})の両辺を加えることで
\begin{align*}
\frac{2k}{L}\{f(\beta_{\ast})-f(\beta_k)\}&\geq \|\beta_{\ast}-\beta_k\|_2^2-\|\beta_{\ast}-\beta_0\|_2^2+\sum_{t=0}^{k-1}t\|\beta_t-\beta_{t+1}\|_2^2\\
&\geq -\|\beta_{\ast}-\beta_0\|_2^2\\
\therefore f(\beta_k)-f(\beta_{\ast})&\leq \frac{L\|\beta_0-\beta_{\ast}\|_2^2}{2k}
\end{align*}
が得られる.これが求めたかった不等式である.
\end{proof}

ISTAの修正アルゴリズムとして,次を考える;実数列$\{\alpha_t\}$を
\begin{align*}
\alpha_1=1,\quad \alpha_{t+1}:=\frac{1+\sqrt{1+4\alpha_t^2}}{2}
\end{align*}
によって定め,さらに初期値$\beta_0$に対して$\gamma_1=\beta_0$として更新式を
\begin{align*}
\beta_t&\leftarrow p(\gamma_t)\\
\gamma_{t+1}&\leftarrow \beta_t+\frac{\alpha_t-1}{\alpha_{t+1}}(\beta_t-\beta_{t-1})
\end{align*}
によって定める.この手法はFISTA\footnote{Fast Iterative Shrinkage-Thresholding Algorithm}と呼ばれ,得られる列$\{\beta_t\}$は次の命題8を満たす
\begin{itembox}[l]{命題8}
FISTAによって生成された数列$\{\beta_t\}$は,$f$の最小化問題に対する最適解を$\beta_{\ast}$として次式を満たす
$$f(\beta_k)-f(\beta_{\ast})\leq \frac{4L\|\beta_0-\beta_{\ast}\|_2^2}{(k+1)^2}$$
\end{itembox}
証明には以下の補題2,3,4を用いる
\begin{itembox}[l]{補題2}
FISTAで考える列$\{\alpha_k\}$と得られる列$\{\beta_k\}$,及び最適解$\beta_{\ast}$に対して次式が成立する
\begin{align*}
\frac{2}{L}\left\{\alpha_k^2(f(\beta_k)-f(\beta_{\ast}))-\alpha_{k+1}^2(f(\beta_{k+1})-f(\beta_{\ast}))\right\}&\geq t_{k+1}-t_k
\end{align*}
ただし$t_k:=\|\alpha_k\beta_k-(\alpha_k-1)\beta_{k-1}-\beta_{\ast}\|_2^2$とする.
\end{itembox}
\begin{itembox}[l]{補題3}
ある実数$c$が存在して,2つの非負値実数列$\{a_k,b_k\}$が任意の自然数$k\geq 1$について
\begin{align*}
a_k-a_{k+1}\geq b_{k+1}-b_k\;かつ\; a_1+b_1\leq c
\end{align*}
を満たすならば,任意の自然数$k$に対して$a_k\leq c$が成立する.
\end{itembox}
\begin{itembox}[l]{補題4}
FISTAで考える列$\{\alpha_k\}$は任意の自然数$k\geq 1$に対して
$$\alpha_k\geq \frac{k+1}{2}$$
が成立する.
\end{itembox}
\begin{proof}
補題2より
$$\frac{2}{L}\alpha_k^2(f(\beta_k)-f(\beta_{\ast}))-\frac{2}{L}\alpha_{k+1}^2(f(\beta_{k+1})-f(\beta_{\ast}))\geq t_{k+1}-t_k$$
が成立する.ここで,$s_k:=\frac{2}{L}\alpha_k^2(f(\beta_k)-f(\beta_{\ast}))$とすることで上式は
\begin{equation}
\label{fis1}
s_{k}-s_{k+1}\geq t_{k+1}-t_{k}
\end{equation}
と書き直せる.さらに$\{s_k\},\{t_k\}$は共に非負値実数列で(\ref{fis1})式及び
\begin{align*}
s_1+t_1&\leq2\|\beta_0-\beta_{\ast}\|_2^2
\end{align*}
を満たすので,$c=2\|\beta_0-\beta_{\ast}\|_2^2$として補題3を適用すれば
$$s_k\leq c\Leftrightarrow \frac{2}{L}\alpha_k^2(f(\beta_k)-f(\beta_{\ast}))\leq 2\|\beta_0-\beta_{\ast}\|_2^2$$
が得られる.上式に補題4を適用すれば,任意の自然数$k$に対して
\begin{align*}
f(\beta_k)-f(\beta_{\ast})&\leq L\|\beta_0-\beta_{\ast}\|_2^2\cdot \frac{1}{\alpha_k^2}\\
&\leq L\|\beta_0-\beta_{\ast}\|_2^2\cdot \frac{4}{(k+1)^2}=\frac{4L\|\beta_0-\beta_{\ast}\|_2^2}{(k+1)^2}
\end{align*}
が成立することがわかる.これが求めたかった不等式である.
\end{proof}


\end{document}