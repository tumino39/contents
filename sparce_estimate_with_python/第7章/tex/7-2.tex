\documentclass{jsarticle}
\usepackage{amsmath,amssymb}
\usepackage{enumerate}
\usepackage{breqn}
\usepackage{amsthm}
\theoremstyle{definition}
\usepackage{bm}
%\newtheorem{Ex}{演習問題}
%\newtheorem*{Ex*}{演習問題}
\newtheoremstyle{mystyle}%   % スタイル名
    {}%                      % 上部スペース
    {}%                      % 下部スペース
    {\normalfont}%           % 本文フォント
    {}%                      % インデント量
    {\bf}%                   % 見出しフォント
    {}%                      % 見出し後の句読点, '.'
    { }%                     % 見出し後のスペース, ' ' or \newline
    {\underline{\thmname{#1}\thmnumber{#2}\thmnote{(#3)}}}%
                             % 見出しの書式 (can be left empty, meaning `normal')
\theoremstyle{mystyle} % スタイルの適用
\newtheorem*{Def}{Def}
\newtheorem*{theo}{Theorem}
\newtheorem*{lem}{Lemma}
\newtheorem*{ex}{Example}
\newtheorem*{col}{Corollary}
\newcommand{\argmax}{\mathop{\rm argmax}\limits}
\newcommand{\argmin}{\mathop{\rm argmin}\limits}
\renewcommand{\footnotesize}{\normalsize}
\usepackage{latexsym}
\def\qed{\hfill$\Box$}
\begin{document}


\large   第1主成分だけでなく第$m$主成分までの最適化をはかる場合,
\begin{align}
\label{tachi}
    L:=\frac{1}{N}\sum_{i=1}^N\|x_i -x_iV_mU_m^T\|_2^2 +\lambda_1\sum_{j=1}^m\|v_j\|_1
            +\lambda_2\sum_{j=1}^m\|v_j\|_2 + \mu\sum_{j=1}^m (\mu_j^T\mu_j -1)
\end{align}
の最小化を考えることになる.ここで,
$$V_m U_m^T = \sum_{j=1}^m v_ju_j^T$$
であることから(\ref{tachi})式はp214(7.10)式第1項の$vu^T$を第$m$項までの和として拡張し,各$v_i$に正則化を施したものと見ることができる.教科書では直交条件$u_ju_k^T=0$が含まれていないが、この条件を含めた定式化ができる.\\

(\ref{tachi})式に直交条件を加えた最小化は
\begin{equation}
    \min_{u,v}\left\{\frac{1}{N}\sum_{i=1}^N\|x_i - x_i V_m U_m^T \|_2^2+\lambda_{1}\sum_{j=1}^m\|v_j\|_1 +\lambda_{2}\sum_{j=1}^m\|v_j\|_2^2\right\}\quad \text{ subject to}\; U_m^T U_m = I_p
\end{equation}
となる.ただし$I_p$は$p$次単位ベクトルとした.上式は双凸であるものの凸ではないので,$U_m,V_m$が与えられたときにそれぞれ$V_m,U_m$を求めるアルゴリズムを考える.\\

\subsection*{$U_m$が与えられたとき}
最小化の式は
\begin{align*}
    \frac{1}{N}\sum_{i=1}^N \|x_i - x_i V_m U_m^T\|_2^2+\lambda_{1}\sum_{j=1}^m\|v_j\|_1 +\lambda_{2}\sum_{j=1}^m\|v_j\|_2^2
\end{align*}
と書くことができ,上式第1項は凸関数であることより上式全体はelastic netの定式化
となっている.したがって,座標降下法により効率的な最小化が図れる.\\

\subsection*{$V_m$が与えられたとき}
最小化の式は
\begin{align}
\label{pro}
    \frac{1}{N}\sum_{i=1}^N \|x_i - x_i V_m U_m^T\|_2^2\quad \text{subject to}\;U_m^T U_m = I_p
\end{align}
と書くことができる.この定式化はプロクルステス問題に書きかえることができ,その書きかえの元で最小化を求める.//

(\ref{pro})式は
\begin{equation}
\tag*{(3)'} 
\label{pro2}
\frac{1}{N}\|X-XV_mU_m^T\|_F^2\quad \text{subject to}\;U_m^T U_m = I_p
\end{equation}
と書き換えることができる.ここで,$\|・\|_F$は行列のフロベニウスノルム,すなわち行列$A=(A)_{i,j}$に対して
$$\|A\|_F=\sqrt{\sum_{i,j}(A)_{i,j}}$$
で定義されるノルムである.また,同じサイズの行列$A,B$に対して,その内積を   
\begin{align*}
    \langle A,B  \rangle:= {\rm tr}(A B^T)
\end{align*}
によって定義する.この定義はwell-definedであり,さらにこの内積から誘導されるノルムは
\begin{align*}
   \langle A,A \rangle={\rm tr}(A A^T)={\rm tr}(A^T A)=\|A\|_F^2
\end{align*}
よりフロベニウスノルムであることがわかる.\\

以上の準備のもと,条件式の最小化について変形していくと
\begin{equation}
\begin{split}
\label{pro3}
\argmin_{U_m}\|X-XV_mU_m^T\|_F^2&=\argmin_{U_m}\langle X-XV_mU_m^T,X-XV_mU_m^T \rangle\\
&=\argmin_{U_m}\left\{\|X\|_F^2+\|XV_mU_m\|_F^2-2\langle X, XV_mU_m^T \rangle
\right\}\\
&=\argmin_{U_m}\left\{\|XV_mU_m^T\|_F^2-2\langle X, XV_mU_m^T \rangle
\right\}
\end{split}
\end{equation}
となり,
\begin{align*}
\|XV_mU_m\|_F^2&={\rm tr}\{XV_mU_m^T(XV_mU_m^T)^T\}\\
&={\rm tr}(XV_mU_m^TU_mV_m^TX^T)\\
&={\rm tr }(XV_m(XV_m)^T)=\|XV_m\|_F^2\quad (\because U_m^TU_m = I_p)
\end{align*}
であることから$\|XV_mU_m\|_F^2$は$U_m$に依存しないことがわかる.したがって(\ref{pro2})式は
\begin{align*}
\argmax_{U_m}\langle X, XV_mU_m^T \rangle
\end{align*}
となることがわかる.さらに変形を進めていくと


\begin{equation}
\begin{split}
\argmax_{U_m}\langle X, XV_mU_m^T \rangle&=\argmax_{U_m}{\rm tr}(X(XV_mU_m^T )^T)\\
&=\argmax_{U_m}{\rm tr}(XU_m V_m^T X^T)\\
&=\argmax_{U_m}{\rm tr}(U_m V_m^T X^TX)
\end{split}
\end{equation}
となり,$V_m^TX^TX$の特異値分解を$V_m^TX^TX=A\Sigma B^T\; (A\in\mathbb{R}^{m\times m},\Sigma \in\mathbb{R}^{m\times p},B\in\mathbb{R}^{p\times p})$とすると上式は
\begin{align*}
\argmax_{U_m}{\rm tr}( U_m A\Sigma B^T)&=\argmax_{U_m}{\rm tr}(B^TU_m A\Sigma)
\end{align*}
とかくことができる.ここで$Z:=B^TU_m A$とおくと,
\begin{align*}
Z^TZ &= (B^TU_m A)^TB^TU_m A\\
&=A^TU_m^TBB^TU_m =I_{m}
\end{align*}
であること,すなわち$Z$の各列はノルム1かつ直交していることがわかる.したがって${\rm tr}(Z\Sigma)$が最大となるには,$Z$の各$i$列が標準単位ベクトルになることが要請される.\\

以上より求める$U_m$は
\begin{align*}
Z = B^TU_m A&=P_m\\
\therefore U_m &= B P_m A^T
\end{align*}
と求められることがわかる.\qed\\



SCotLASSと共に,SPCAは目的関数が共に凸ではないが双凸である.さらにSPCAにはelastic net のアルゴリズムを適用でき,かつ複数成分を一度に求められるといった利点がある.


\end{document}





\end{document}
