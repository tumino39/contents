\documentclass{jsarticle}
\usepackage{amsmath,amssymb}
\usepackage{enumerate}
\usepackage{breqn}
\usepackage{amsthm}
\theoremstyle{definition}
\usepackage{bm}
%\newtheorem{Ex}{演習問題}
%\newtheorem*{Ex*}{演習問題}
\newtheoremstyle{mystyle}%   % スタイル名
    {}%                      % 上部スペース
    {}%                      % 下部スペース
    {\normalfont}%           % 本文フォント
    {}%                      % インデント量
    {\bf}%                   % 見出しフォント
    {}%                      % 見出し後の句読点, '.'
    { }%                     % 見出し後のスペース, ' ' or \newline
    {\underline{\thmname{#1}\thmnumber{#2}\thmnote{(#3)}}}%
                             % 見出しの書式 (can be left empty, meaning `normal')
\theoremstyle{mystyle} % スタイルの適用
\newtheorem*{Def}{Def}
\newtheorem*{theo}{Theorem}
\newtheorem*{lem}{Lemma}
\newtheorem*{ex}{Example}
\newtheorem*{col}{Corollary}
\renewcommand{\footnotesize}{\normalsize}
\usepackage{latexsym}
\def\qed{\hfill$\Box$}


\begin{document}
\section*{主成分分析}
\vspace{5cm}
\large 以下,行列$X=(x_{i,j})\in \mathbb{R}^{N\times p}$は中心化されていると仮定する.すなわち,各$j=1,\cdots ,p$に対して
\begin{align*}
\sum_{i=1}^Nx_{i,j}=0
\end{align*}
であるとする.$\|v\|_2^2=1$のもとで,
\begin{align}
\label{shusei}
\|Xv\|_2^2=v^TX^TXv
\end{align}
を最大にする$v$を$v_1$,(\ref{shusei})式を最大にして$v_1$と直交する$v$を$v_2,\cdots$というようにして正規直交系$V:=[v_1,\cdots,v_p]$を求める操作を主成分分析という.\\
\vspace{7cm}

まず各$v_j$が直交するという制約を除いて,$\|v\|_2^2=1$のもとで$\|Xv\|_2^2$を最大にする$v_j$について考えてみる.そのような$v_j$は
$$L(v_j,\mu_j)=\|Xv_j\|_2^2-\mu_j(\|v_j\|_2^2-1)$$
を最大にするので,上式を$v_j$で微分して$0$とおいた等式
$$X^TXv_j-\mu_j v_j=0$$
を満足する.上式を,$X$の標本共分散行列$\Sigma :=\frac{1}{N}X^TX$および$\lambda:=\frac{\mu_j}{N}$を用いて書き換えると
$$\Sigma v_j = \lambda_j v_j$$
と書くことができる.これより求める$v_j$は$\Sigma$の固有ベクトルで,$\lambda_j$は$v_j$が属する固有値になることがわかる.もし$\lambda_j$の中で重複度が2以上のものだある場合には,それらの固有ベクトルは直交するように選んでくる.$\Sigma$の固有値が全てことなる場合には,$v_1,\cdots,v_p$は自動的に直交することがいえる.\\
\vspace*{5cm}

実際には$v_1,\cdots,v_p$を全て用いることはなく,最初の$m$個のみを用いることになる.そして$X$の各行を$V_m:=[v_1,\cdots,v_m]$に射影した$Z:=XV_m$を得る.すなわち$p$次元の情報を$m$個の主成分$v_1,\cdots,v_m$の空間に射影して,$m$次元の$Z$で$p$次元の$X$を見ることになる.そのような次元の圧縮のための線形写像が,主成分分析である.\\
\vspace*{3cm}

主成分分析のスパースなアプローチにはいくつかあるので,それらを考察していく.まず非ゼロ要素の個数を制限する手法について,この場合は$t$を整数として,$\|v\|_0\leq t,\| v\|_2=1$のもとで
$$v^TX^TXv-\lambda \|v\|_0$$
を最大化するような定式化になる.しかしこの場合,目的関数が凸にはならない.\\

また,$\|v\|_1\leq t(t>0)$の制約を持たせて,$\|v\|_2=1$のもとで
\begin{equation}
\label{l1}
v^TX^TXv-\lambda \|v\|_1
\end{equation}
の最大化を図ろうとしても,目的関数は凸にならない.\\
\vspace{5cm}

そこで$u\in\mathbb{R}^N$として,$\|u\|_2=\|v\|_2=1$のもとで
\begin{equation}
\label{SCoT}
u^TXv-\lambda\|v\|_1
\end{equation}の最大化を図る定式化,SCoTLASS\footnote{Simplified Component Technique - LASSO}が提案された.(\ref{SCoT})式で得られる最適な$v$は,(\ref{l1})の最適解になっている.実際
\begin{align}
\label{rl}
L:=-u^TXv+\lambda\|v\|_1+\frac{\mu}{2}(u^Tu-1)+\frac{\delta}{2}(v^Tv-1)
\end{align}
を$u$で偏微分して$0$とおくと
\begin{align*}
\frac{\partial L}{\partial u}=X v+\mu u=0
\end{align*}
となり,$\|u\|_2^2=1$であることから$\displaystyle u = \frac{Xv}{\|Xv\|_2}$となる.これを(\ref{rl})式に代入することで
\begin{align}
-\|Xv\|_2+\lambda\|v\|_1+\frac{\delta}{2}(v^Tv-1)
\end{align}
となる.






\end{document}
