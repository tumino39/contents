\documentclass{jsarticle}
\usepackage{amsmath,amssymb}
\usepackage{enumerate}
\usepackage{breqn}
\usepackage{amsthm}
\theoremstyle{definition}
\usepackage{bm}
%\newtheorem{Ex}{演習問題}
%\newtheorem*{Ex*}{演習問題}
\newtheoremstyle{mystyle}%   % スタイル名
    {}%                      % 上部スペース
    {}%                      % 下部スペース
    {\normalfont}%           % 本文フォント
    {}%                      % インデント量
    {\bf}%                   % 見出しフォント
    {}%                      % 見出し後の句読点, '.'
    { }%                     % 見出し後のスペース, ' ' or \newline
    {\underline{\thmname{#1}\thmnumber{#2}\thmnote{(#3)}}}%
                                % 見出しの書式 (can be left empty, meaning `normal')
\theoremstyle{mystyle} % スタイルの適用
\newtheorem*{Def}{Def}
\newtheorem*{theo}{Theorem}
\newtheorem*{lem}{Lemma}
\newtheorem*{ex}{Example}
\newtheorem*{col}{Corollary}
\renewcommand{\footnotesize}{\normalsize}
\usepackage{latexsym}
\def\qed{\hfill$\Box$}

\begin{document}
\Large
確率変数$Y$が実数$\mu$を用いて
$$P(Y=k)=\frac{\mu^k}{k!}e^{-\mu}$$
で表される分布に従うとき、$Y$はポアソン分布に従うという.このとき
\begin{align*}
E[Y]&=\sum_{i=0}^{\infty}k\cdot \frac{\mu^k}{k!}e^{-\mu}\\
&=\sum_{i=1}^{i\infty}\frac{\mu\cdot \mu^{k-1}}{(k-1)!}e^{-\mu}\\
&=\mu e^{-\mu}\sum_{l=0}^{\infty}\frac{\mu^l}{l!}=\mu
\end{align*}
となることから$Y$の期待値は$\mu$になることがわかる.\\

ポアソン分布のパラメータ$\mu$に関して,$p$を自然数とし$x\in \mathbb{R}^p$に対してある$\beta_0\in \mathbb{R},\beta\in \mathbb{R}^p$が存在して
$$\mu(x):=E[Y\mid X=x]=e^{\beta_0+x\beta}$$
とかけることを仮定する.この場合$N$個の観測値$(x_1,y_1),\cdots ,(x_N,y_N)\;(x_i\in \mathbb{R}^p,y_i\in\mathbb{Z}_+)$についての尤度関数は
$$\mu_i:=\mu(x_i)=e^{\beta_0+x_i\beta}$$
とすれば
\begin{align}
\prod_{i=1}^N\frac{{\mu_i}^{y_i}}{y_i!}e^{-\mu_i}
\end{align}
とかくことができる.各$i$に対して
\begin{align*}
\log\frac{{\mu_i}^{y_i}}{y_i!}e^{-\mu_i}&=y_i\log \mu_i+\log e^{-\mu_i}-\log(y_i!)\\
&=y_i(\beta_0+x\beta)-e^{\beta_0+x\beta}-\log(y_i!)
\end{align*}
が成立していることより,(1)式のマイナス対数をとると
\begin{align*}
-\log\left(\prod_{i=1}^N\frac{{\mu_i}^{y_i}}{y_i!}e^{-\mu_i}\right)&=-\sum_{i=1}^N\log\left(frac{{\mu_i}^{y_i}}{y_i!}e^{-\mu_i}\right)\\
&=-\sum_{i=1}^N\{y_i(\beta_0+x_i\beta)-e^{\beta_0+x_i\beta}-\log(y_i!)\}\\
&=-\sum_{i=1}^N\{y_i(\beta_0+x_i\beta)-e^{\beta_0+x_i\beta}\}+\sum_{i=1}^N\log(y_i!)
\end{align*}
となることがわかる.したがって(1)式のマイナス対数の($\beta_0,\beta$に関する)最小化は上式の最終式第1項の最小化と同値になるのでこれにLassoを適用することを考える.すなわち
\begin{align*}
L(\beta_0,\beta)=-\frac{1}{N}\sum_{i=1}^N\{y_i(\beta_0+x_i\beta)-e^{\beta_0+x_i\beta}\}
\end{align*}
とし,これに正則化項を付けた
\begin{align*}
L(\beta_0,\beta)+\lambda\|\beta\|
\end{align*}
の最小化について考える.\\

ロジスティック回帰の場合と同様にニュートン法を用いることを考える.そのために$x_i=(x_{i,1},\cdots ,x_{i,p})^T$とし,
$$X=\begin{pmatrix}
1 & x_{1,1} & \cdots & x_{1,p}\\
\vdots & \vdots &\ddots & \vdots\\
1 & x_{N,1}&\cdots & x_{N,p}
\end{pmatrix}$$
としたうえで

$\nabla L=-\frac{1}{N}Xu,\nabla^2 L=\frac{1}{N}X^TWX$を満たす$u\in \mathbb{R}^{N},W\in \mathbb{R}^{N\times N}$を求める.\\

$L$の$\beta_j(j=0,\cdots,p)$での偏微分を考えると,$x_{i,j}$を$x_i\in \mathbb{R}^p$の第$j$成分として
\begin{align*}
\frac{\partial L}{\partial \beta_0}&=-\frac{1}{N}\sum_{i=1}^N(y_i-e^{\beta_0+x_i\beta})\\
\frac{\partial L}{\partial \beta_j}&=-\frac{1}{N}\sum_{i=1}^N(x_{i,j}-x_{i,j}e^{\beta_0+x_i\beta})\\
&=-\frac{1}{N}\sum_{i=1}^Nx_{i,j}(y_i-e^{\beta_0+x_i\beta})
\end{align*}
と書くことができるので,$\nabla L$は
\begin{align*}
\nabla L&=-\frac{1}{N}\left(\begin{array}{c}
\sum_{i=1}^N(y_1-e^{\beta_0+x_i\beta_i})\\
\sum_{i=1}^Nx_{i,1}(y_i-e^{\beta_0+x_i\beta_i})\\
\vdots \\
\sum_{i=1}^Nx_{i,p}(y_i-e^{\beta_0+x_i\beta_i})
\end{array}\right)\\
&=-\frac{1}{N}\begin{pmatrix}
1 & x_{1,1} & \cdots & x_{1,p}\\
\vdots & \vdots &\ddots & \vdots\\
1 & x_{N,1}&\cdots & x_{N,p}
\end{pmatrix}^T\begin{pmatrix}
y_1-e^{\beta_0+x_1\beta}\\
\vdots\\
y_N-e^{\beta_0+x_N\beta}\\
\end{pmatrix}
\end{align*}
となり,したがって
$$u=\begin{pmatrix}
y_1-e^{\beta_0+x_1\beta}\\
\vdots\\
y_N-e^{\beta_0+x_N\beta}\\
\end{pmatrix}$$
とすればよいことがわかる.\\

また$L$の二階微分を考えると,$x_{i,0}=1(i=1,\cdots ,N)$として
\begin{align*}
\frac{\partial L^2}{\partial \beta_j \beta_k}=\frac{1}{N}\sum_{i=1}^Nx_{i,j}x_{i,k}e^{\beta_0+x_i\beta}\quad (j,k\in\{0,1,\cdots,p\})
\end{align*}
となることより

\begin{align*}
\nabla^2 L&=\frac{1}{N}\begin{pmatrix}
\sum_{i=1}^Nx_{i,1}x_{i,1}e^{\beta_0+x_i\beta} & \cdots & \sum_{i=1}^Nx_{i,1}x_{i,p}e^{\beta_0+x_i\beta}\\
\vdots & \ddots & \vdots \\
\sum_{i=1}^Nx_{i,N}x_{i,1}e^{\beta_0+x_i\beta} & \cdots & \sum_{i=1}^Nx_{i,N}x_{i,P}e^{\beta_0+x_i\beta}
\end{pmatrix}\\
&=\frac{1}{N}\begin{pmatrix}
1 & x_{1,1} & \cdots & x_{1,p}\\
\vdots & \vdots &\ddots & \vdots\\
1 & x_{N,1}&\cdots & x_{N,p}
\end{pmatrix}^T\begin{pmatrix}
e^{\beta_0+x_1\beta}& e^{\beta_0+x_1\beta}x_{1,1} & \cdots & e^{\beta_0+x_1\beta}x_{1,p}\\
\vdots & \vdots &\ddots & \vdots\\
e^{\beta_0+x_N\beta}& e^{\beta_0+x_N\beta}x_{N,1}&\cdots & e^{\beta_0+x_N\beta}x_{N,p}
\end{pmatrix}\\
&=\frac{1}{N}\begin{pmatrix}
1 & x_{1,1} & \cdots & x_{1,p}\\
\vdots & \vdots &\ddots & \vdots\\
1 & x_{N,1}&\cdots & x_{N,p}
\end{pmatrix}^T\begin{pmatrix}
e^{\beta_0+x_1\beta} & \cdots & 0\\
\vdots  & \ddots & \vdots\\
0 & \cdots & e^{\beta_0+x_N\beta}
\end{pmatrix}\begin{pmatrix}
1 & x_{1,1} & \cdots & x_{1,p}\\
\vdots & \vdots &\ddots & \vdots\\
1 & x_{N,1}&\cdots & x_{N,p}
\end{pmatrix}\\
\end{align*}
となり,したがって
$$W=\begin{pmatrix}
e^{\beta_0+x_1\beta} & \cdots & 0\\
\vdots  & \ddots & \vdots\\
0 & \cdots & e^{\beta_0+x_N\beta}
\end{pmatrix}$$
とすればよいことがわかる.





\end{document}
