\documentclass{jsarticle}
\usepackage{amsmath,amssymb}
\usepackage{enumerate}
\usepackage{breqn}
\usepackage{amsthm}
\theoremstyle{definition}
\usepackage{bm}
%\newtheorem{Ex}{演習問題}
%\newtheorem*{Ex*}{演習問題}
\newtheoremstyle{mystyle}%   % スタイル名
    {}%                      % 上部スペース
    {}%                      % 下部スペース
    {\normalfont}%           % 本文フォント
    {}%                      % インデント量
    {\bf}%                   % 見出しフォント
    {}%                      % 見出し後の句読点, '.'
    { }%                     % 見出し後のスペース, ' ' or \newline
    {\underline{\thmname{#1}\thmnumber{#2}\thmnote{(#3)}}}%
                             % 見出しの書式 (can be left empty, meaning `normal')
\theoremstyle{mystyle} % スタイルの適用
\newtheorem*{Def}{Def}
\newtheorem*{theo}{Theorem}
\newtheorem*{lem}{Lemma}
\newtheorem*{ex}{Example}
\newtheorem*{col}{Corollary}
\renewcommand{\footnotesize}{\normalsize}
\title{正定値行列のCholesky分解可能性に関する証明}
\date{}
\usepackage{latexsym}
\usepackage{emathEy}
\def\qed{\hfill$\Box$}

\begin{document}
\maketitle
\Large
$N$次実対称行列$A\in S_N(\mathbb{R})$を正定値行列とする.すなわち,$A$は任意の${\bm x}\in \mathbb{R}^N$に対し,
$${\bm x}\neq 0\Rightarrow {\bm x}^T A{\bm x}>0$$
を常に満たすような行列とする.以下では,ある上三角行列$M$が存在して
$$A=M^TM$$
とかけることをいくつかのステップに分けて証明する.\\


以下の証明では,正定値行列$A=\displaystyle\begin{pmatrix}
a_{1,1}&\cdots&a_{1,N}\\
\vdots & \ddots & \vdots \\
a_{1,N} & \cdots & a_{N,N} 
\end{pmatrix}$と各$i\in \{2,\cdots,N-1\}$に対し,ベクトル${\bm a}_i\in \mathbb{R}^i$と対称行列$A_i\in S_i(\mathbb{R})$をそれぞれ
\begin{align*}
{\bm a}_i &:=\left(\begin{array}{c}
a_{1,i}\\
a_{2,i}\\
\vdots\\
a_{i-1,i}
\end{array}\right),\quad A_1 :=(a_{1,1}),\quad A_i:=\begin{pmatrix}
A_{i-1} & {\bm a}_i\\
{\bm a}_i^T & a_{i,i}
\end{pmatrix}
\end{align*}
によって定義する.また,正定値行列の固有値は全て正であることに注意する.\\
さらに行列$Y\in \mathbb{R}^{N\times N}$が$U^TDU$分解可能であるということを,ある単位上三角行列$U$,対角行列$D$を用いて
$$Y=U^TDU$$
の形に分解できることとして定義する.\\

\subsection*{\Large $A_i$が正則行列,特に正定値行列であること}
対角行列$P_i\in \mathbb{R}^{N\times i}$を
$$P_i:=({\bm e_1},{\bm e_2},\cdots ,{\bm e_i})$$
によって定義する.ここで,各$i$に対し$e_i$を$\mathbb{R}^N$における$i$番目の標準基底とした.このとき,
\begin{align*}
P_i^TA P_i&=\left(\begin{array}{ccc}
1 & \cdots & O \\
\vdots & \ddots & \vdots \\
O & \cdots & 1 \\
\hline
{\bm 0} & \cdots & {\bm 0}
\end{array}\right)^T
\begin{pmatrix}
a_{1,1}&\cdots&a_{1,N}\\
\vdots & \ddots & \vdots \\
a_{N,1} & \cdots & a_{N,N} 
\end{pmatrix}\left(\begin{array}{ccc}
1 & \cdots & O \\
\vdots & \ddots & \vdots \\
O & \cdots & 1 \\
\hline
{\bm 0} & \cdots & {\bm 0}
\end{array}\right)\\
&=\left(\begin{array}{ccc}
1 & \cdots & O \\
\vdots & \ddots & \vdots \\
O & \cdots & 1 \\
\hline
{\bm 0} & \cdots & {\bm 0}
\end{array}\right)^T\begin{pmatrix}
a_{1,1}&\cdots & a_{1,i}\\
\vdots & \ddots & \vdots \\
a_{N,1} & \cdots & a_{N,i}
\end{pmatrix}\\
&=\begin{pmatrix}
a_{1,1}&\cdots & a_{1,i}\\
\vdots & \ddots & \vdots \\
a_{i,1} & \cdots & a_{i,i}
\end{pmatrix}=A_i
\end{align*}
となるので
$$A_i = P_i^T AP$$
が成立することがわかる.さらに,$P_i$の各列の線形独立性より${\rm rank}\;P_i = i$であることがわかるので,これと線形写像の次元に関する定理から
$${\rm null}P_i = i-{\rm rank}P_i=0$$
であることがわかり,したがって${\rm Ker}P_i= \{{\bm 0}\}$であることがわかる.よって任意の${\bm x}\in \mathbb{R}^i$に対して
\begin{align*}
{\bm x}\neq {\bm 0}&\Rightarrow P_i{\bm x}\neq {\bm 0}\\
&\Rightarrow (P_i{\bm x})^T A(P_i{\bm x})={\bm x}^TP_i^TAP_i{\bm x}>0\\
&\Rightarrow {\bm x}^TA_i{\bm x}>0
\end{align*}
が常に成立するので,$A_i$が正定値行列であることがわかる.\\

\subsection*{\Large $A_2$が$U^TDU$分解可能であること}
各$i\in \{1,\cdots ,N\}$に対し,$a_{i,i}\neq 0$であることに注意する.実際$a_{i,i}=0$であるとすると,\\
$\displaystyle {\bm e}_i\in \mathbb{R}^N$に対して
\begin{align*}
{{\bm e}_i}^TA{\bm e}_i&=a_{i,i}=0
\end{align*}
となるが,これは$A$が正定値であることに矛盾する.したがって特に$a_{1,1}\neq 0$であるので,これを踏まえた上で,実数$u,k\in \mathbb{R}$を
$$u:= \frac{a_{1,2}}{a_{1,1}},\quad k:=a_{2,2}-a_{1,1}u^2$$
で定めれば
\begin{align*}
\begin{pmatrix}
1 & u\\
0 & 1
\end{pmatrix}^T\begin{pmatrix}
a_{1,1} & 0\\
0 & k
\end{pmatrix}\begin{pmatrix}
1 & u\\
0 & 1
\end{pmatrix}&=\begin{pmatrix}
1 & 0\\
u & 1
\end{pmatrix}\begin{pmatrix}
a_{1,1} & a_{1,1}u\\
0 & k
\end{pmatrix}\\
&=\begin{pmatrix}
a_{1,1} & a_{1,1}u\\
a_{1,1}u & k+a_{1,1}u^2
\end{pmatrix}\\
&=\begin{pmatrix}
a_{1,1} & a_{1,2}\\
a_{1,2} & a_{2,2}
\end{pmatrix}=A_2
\end{align*}
となるので
$$A_2=\begin{pmatrix}
1 & u\\
0 & 1
\end{pmatrix}^T\begin{pmatrix}
a_{1,1} & 0\\
0 & k
\end{pmatrix}\begin{pmatrix}
1 & u\\
0 & 1
\end{pmatrix}$$
とかけること,すなわち$A_2$が$U^TDU$分解可能であることがわかる.\\

\subsection*{\Large $A_{i-1}$が$U^TDU$分解可能なら$A_i$も$U^TDU$分解可能であること}
$A_{i-1}$が$U^TDU$分解可能であるとする.すなわちある単位上三角行列$U_{\ast}\in \mathbb{R}^{i\times i}$と対角行列$D_{\ast}\in \mathbb{R}^{i\times i}$が存在して
$$A_{i-1}=U_{\ast}^TD_{\ast}U_{\ast}.$$
ここで$A_{i-1}$は正定値行列で特に正則行列であることより,あるベクトル${\bm r}\in \mathbb{R}^{i-1}$が存在して
\begin{align*}
A_{i-1}{\bm r}={\bm a}_{i-1}
\end{align*}
が成立することがわかる.したがってベクトル${\bm u}\in \mathbb{R}^{i-1}$を${\bm u}:=U_{\ast}{\bm r}$によって定義すれば
$$U_{\ast}^T D_{\ast}{\bm u}=U_{\ast}^T D_{\ast}U_{\ast}{\bm r}=A_{i-1}{\bm r}={\bm a}_{i-1}$$
が成立し,さらに
$$k:=a_{i,i}-{\bm u}^TD_{\ast} {\bm u}$$
とすることで
\begin{align*}
\begin{pmatrix}
U_{\ast}  & {\bm u}\\
{\bm 0}^T  & 1
\end{pmatrix}^T \begin{pmatrix}
D_{\ast} &{\bm 0}\\
{\bm 0}^T & k
\end{pmatrix}\begin{pmatrix}
U_{\ast}  & {\bm u}\\
{\bm 0}^T  & 1
\end{pmatrix}&=\begin{pmatrix}
U_{\ast}^T  & {\bm 0}\\
{\bm u}^T  & 1
\end{pmatrix}\begin{pmatrix}
D_{\ast}U_{\ast} & D_{\ast}{\bm u}\\
{\bm 0}^T & k
\end{pmatrix}\\
&=\begin{pmatrix}
U_{\ast}^TD_{\ast}U_{\ast} & U_{\ast}^TD_{\ast}{\bm u}\\
{\bm u}^TD_{\ast}U_{\ast} & {\bm u}^TD_{\ast}{\bm u}+k
\end{pmatrix}\\
&=\begin{pmatrix}
A_{i-1} & {\bm a}_{i-1}\\
{\bm u}^T & a_{i,i}
\end{pmatrix} = A_i
\end{align*}
が成立することから
\begin{align*}
A_i &=\begin{pmatrix}
U_{\ast}  & {\bm u}\\
{\bm 0}^T  & 1
\end{pmatrix}^T \begin{pmatrix}
D_{\ast} &{\bm 0}\\
{\bm 0}^T & k
\end{pmatrix}\begin{pmatrix}
U_{\ast}  & {\bm u}\\
{\bm 0}^T  & 1
\end{pmatrix}
\end{align*}
と$A_i$が$U^TDU$分解できることがわかる.これと$A_2$が$U^TDU$分解できることより,$i\in \{2,\cdots N\}$に対して$A_i$が$U^TDU$分解できることがわかる.\\

\subsection*{\Large $D$の対角成分が全て正であること}
先の議論で特に$i=N$とすれば,正定値$A=A_N$が$U^TDU$分解できることがわかるので,その分解を与える単位上三角行列,対角行列をそれぞれ$U,D$とする.すなわち
$$A=U^TDU$$
さらに$U$は単位上三角行列であったことから,各列の線形独立性より$U$は正則行列であることに注意する.\\

対角行列$D$は
\begin{align*}
D&=(U^T)^{-1}(U^T D U )(U^{-1})=(U^{-1})^TAU^{-1}
\end{align*}
と変形することができる.また,$U^{-1}$が正則行列であること,すなわち
$$ {\rm Ker}\;U^{-1}=\{{\bm 0}\}$$
であることと$A$が正定値であることより,$(U^{-1})^TAU^{-1}$が正定値となることがわかるので$D$も正定値となることがわかる.これと正定値行列の対角成分が全て正であることより,$D$の対角成分が全て正であることがわかる.\\

以上より$A$は単位上三角行列$U$と成分が全て正である対角行列$D$を用いて
$$A=U^T DU$$
と分解できることが示せた.ここで$U$の対角成分をそれぞれ$d_1,\cdots,d_N$とし,
$$D^{1/2}:=\begin{pmatrix}
\sqrt{d_1} &  & O\\
 & \ddots & \\
O & & \sqrt{d_N}
\end{pmatrix}$$
によって定義すると$D=D^{1/2}D^{1/2}$が成立し,さらに
\begin{align*}
A&=U^T DU=U^T (D^{1/2})^T D^{1/2}U=(D^{1/2} U)^T (D^{1/2}U)
\end{align*}
とかけることがわかる.$D^{1/2}U$は上三角行列であるので,改めて$M=D^{1/2}U$とすれば$A$は上三角行列$M$を用いて
$$A=M^T M$$
とかけること,すなわち$A$がCholesky分解可能であることが示せた.\qed


\subsection*{\Large 分解の一意性に関する補足}
正定値$A$のCholesky分解$A=M^TM$が一意であることを示す;$A$の$U^TDU$分解$A=U^TDU$に対し,ある単位上三角行列$U'$と$D'$が存在して$A={U'}^TD'U'$を満たしたとする.このとき
\begin{align}
\label{ponpon}
({U'}^T)^{-1}{U^T}D&=({U'}^T)^{-1}(U^TDU)U^{-1}=({U'}^T)^{-1}({U'}^TD'U')U^{-1}=D'U'U^{-1}
\end{align}
が成立することがわかる.ここで$({U'}^T)^{-1},U^T$がどちらも単位下三角行列であることより積$({U'}^T)^{-1},U^T$の対角成分は全て$1$となり,したがって$({U'}^T)^{-1}{U^T}D$の対角成分はすべて$D$の対角成分に等しい.\\
同様に$U',U^{-1}$はどちらも単位上三角行列であることより積$U'U^{-1}$の対角成分は全て$1$であり,したがって$D'U'U^{-1}$の対角成分は全て$D'$の対角成分に等しいことがわかる.\\

以上と等式(\ref{ponpon})より,$D=D'$であることがまずわかる.\\


さらに,等式
$$U^TDU={U'}^TD'U'$$
の両辺に左から$({U'}^T)^{-1}$,右から$U^{-1}D^{-1}$をかけることで等式
$$({U'}^T)^{-1}U^T=D'U'U^{-1}D^{-1}$$
が得られ,上式の左辺は単位下三角行列,右辺は単位上三角行列であることから
$$({U'}^T)^{-1}U^T=D'U'U^{-1}D^{-1}=U'U^{-1}=E$$
とかけることがわかる.ただし$E$は単位行列とした.\\

以上と逆行列の一意性より
\begin{align*}
 {U'}^{-1}&=U^{-1}\\
\therefore U'&=U
\end{align*}
がわかるので,これよりCholesky分解の一意性がわかる.\qed


\end{document}