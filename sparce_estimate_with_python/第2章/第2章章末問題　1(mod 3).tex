\documentclass{jsarticle}
\usepackage{amsmath,amssymb}
\usepackage{enumerate}
\usepackage[dvipdfmx]{graphicx}
\usepackage[dvipdfmx]{color}
\usepackage{here}
\usepackage{breqn}
\usepackage{amsthm}
\theoremstyle{definition}
\usepackage{bm}
\newtheorem{Ex}{問題}
%\newtheorem*{Ex*}{演習問題}
\newtheoremstyle{mystyle}%   % スタイル名
    {}%                      % 上部スペース
    {}%                      % 下部スペース
    {\normalfont}%           % 本文フォント
    {}%                      % インデント量
    {\bf}%                   % 見出しフォント
    {}%                      % 見出し後の句読点, '.'
    { }%                     % 見出し後のスペース, ' ' or \newline
    {\underline{\thmname{#1}\thmnumber{#2}\thmnote{(#3)}}}%
                             % 見出しの書式 (can be left empty, meaning `normal')
\theoremstyle{mystyle} % スタイルの適用
\newtheorem*{Def}{Def}
\newtheorem*{theo}{Theorem}
\newtheorem*{lem}{Lemma}
\newtheorem{ex}{example}
\newtheorem*{col}{Corollary}
\renewcommand{\footnotesize}{\normalsize}
\usepackage{latexsym}
\usepackage{color}
\usepackage{emathEy}
\usepackage{listings,plistings}
\def\qed{\hfill$\Box$}
\lstset{
language={Python},
backgroundcolor={\color[gray]{.85}},
basicstyle={\footnotesize},
identifierstyle={\footnotesize},
commentstyle={\footnotesize\ttfamily \color[rgb]{0,0.5,0}},
keywordstyle={\footnotesize\bfseries \color[rgb]{1,0,0}},
ndkeywordstyle={\footnotesize},
stringstyle={\footnotesize\ttfamily \color[rgb]{0,0,1}},
frame={tb},
breaklines=true,
columns=[l]{fullflexible},
numbers=left,
xrightmargin=0zw,
xleftmargin=3zw,
numberstyle={\scriptsize},
stepnumber=1,
numbersep=1zw,
morecomment=[l]{//}
}

\begin{document}
\large
\begin{Ex}
\begin{enumerate}[(a)]
\item 目的関数
$$L(\beta_0,\beta):=\sum_{i=1}^N \log\{1+\exp(-y_i(\beta_0+x_i^T \beta))\}$$
を$\beta_j(j=0,\cdots,p)$で偏微分するとき,$x_{i,0}=1(i=1,\cdots ,N),v_i :=\exp(-y_i(\beta_0+x_i^T \beta)) $と置くことで
\begin{align*}
\frac{\partial L}{\partial \beta_j}&=\sum_{i=1}^N\frac{-y_i\cdot x_{i,j}\exp(-y_i(\beta_0+x_i^T \beta))}{1+\exp(-y_i(\beta_0+x_i^T \beta))}\\
&=-\sum_{i=1}^N\frac{y_ix_{i,j} v_i}{1+v_i}
\end{align*}
とかけることより
\begin{align*}
\nabla L&=-\begin{pmatrix}
1 & \cdots & 1\\
x_{1,1} & \cdots & x_{N,1}\\
\vdots & \ddots & \vdots \\
x_{1,p} & \cdots & x_{N,p}
\end{pmatrix}\left(\begin{array}{c}
\frac{y_1 v_i}{1+v_1}\\
\vdots \\
\frac{y_1 v_N}{1+v_N}
\end{array}\right)=-X^T u
\end{align*}
と書くことができる.\\

\item (a)で求めた$\frac{\partial L}{\partial \beta_j}$をさらに$\beta_k$で偏微分すると
\begin{align*}
\frac{\partial^2 L}{\partial \beta_j \partial \beta_k}&=-\sum_{i=1}^N \left\{\frac{\partial v_i}{\partial \beta_k}\frac{\partial }{\partial v_i}\left(\frac{y_ix_{i,j}v_i}{1+v_i}\right)\right\}\\
&=\sum_{i=1}^Ny_i^2x_{i,j}x_{i,k}\frac{v_i}{(1+v_i)^2}\\
&=\sum_{i=1}^Nx_{i,j}x_{i,k}\frac{v_i}{(1+v_i)^2}\quad(\because y_i\in \{\pm 1\})
\end{align*}
となることより,
\begin{align*}
\nabla^2 L &=X^T\begin{pmatrix}
\frac{v_1}{(1+v_1)^2} & \cdots & 0\\
\vdots & \ddots & \vdots \\
0 & \cdots & \frac{v_N}{(1+v_N)^2}
\end{pmatrix}X\\
&=X^TWX
\end{align*}
と書くことができる.\\
\end{enumerate}
さらに,上記で求めた行列を用いて作成した,Newton法により値を更新していくプログラムの実行結果は下の通り.最上段が初期値($p+1$次元)を,最下段が推定された係数と実際の係数の$l2$ノルムを表す

\begin{lstlisting}[basicstyle = \ttfamily\footnotesize, frame = single]
#p=2
[ 0.63094861 -0.92697651 -0.91072333]
[ 0.86047835 -1.14017869 -1.35275858]
[ 0.95114002 -1.22561286 -1.50788179]
[ 0.96108609 -1.235      -1.52382468]
0.7609464718257709

#p=3
[-0.39248015 -0.60068543 -0.6049099  -1.33818295]
[-0.46124804 -0.72404346 -0.72430081 -1.87481017]
[-0.48976562 -0.77927907 -0.771821   -2.06096778]
[-0.49271985 -0.78517205 -0.77653638 -2.0784206 ]
0.7252842383848371

#p=4
[ 0.08141478 -0.68484112  0.70987333 -1.50021169  0.0729047 ]
[ 0.0575044  -0.92584186  1.0212524  -2.04351767  0.1381232 ]
[ 0.05396007 -1.0284011   1.15461278 -2.28043797  0.16276137]
[ 0.05397496 -1.04263046  1.17324641 -2.31351461  0.16560318]
[ 0.05398124 -1.04286421  1.17355273 -2.31405542  0.16564041]
1.183505858275822

#p=5
[-0.44735853  0.1753804   0.74519156 -0.75696786  0.95888674  0.52116991]
[-0.51760096 -0.84462375  0.78877956 -1.1092826   0.70638379 -0.42738666]
[-0.74273836 -0.84191146  1.08434294 -1.39660604  1.10210054 -0.30626662]
[-0.81581402 -0.91371154  1.18648688 -1.52188011  1.2140128  -0.32811659]
[-0.82315975 -0.9212835   1.19631503 -1.53466277  1.22504241 -0.33062673]
0.8536261412241056

#p=6
[-0.30254961 -0.65441239 -0.7671653   0.89556886  0.82291769 -1.8986078
  1.09200474]
[-0.93939836 -0.85281913  0.71360078  0.24822987  0.12968842  0.23661314
  0.79192553]
[-0.78674208 -0.88892596  0.19398896  0.61617608  0.47178312 -0.73113428
  1.06501684]
[-0.98383111 -1.13192227  0.26832396  0.75029806  0.56505247 -0.8264384
  1.33542463]
[-1.03257514 -1.1857139   0.28516761  0.78091463  0.59207949 -0.84812315
  1.40043966]
[-1.03483699 -1.18803046  0.28591791  0.7822523   0.59338503 -0.8490672
  1.40340034]
1.0691750340249508
\end{lstlisting}
$p$を大きくしていくごとに,シミュレーションの計算が発散してしまう回数が増えていった.\qed\\
\end{Ex}
%%%%%%%%%%%%%%%%%%%%%%%%%%%%%%%%%%%%%%%%%%%%%

\begin{Ex}
\begin{enumerate}[(a)]
\item $\gamma_0\in\mathbb{R},\gamma\in\mathbb{R}^p$を任意にとって固定し,(2.25)式の指数部全てから$\gamma_0+x^T\gamma$を引いた式を変形していくと
\begin{align*}
&\hspace{-3cm}\frac{\exp(\beta_{0,k}+x^T\beta^{(k)}-(\gamma_0+x^T\gamma))}{\sum_{l=1}^K\exp(\beta_{0,k}+x^T\beta^{(l)}-(\gamma_0+x^T\gamma))}\\
&=\frac{\exp(-(\gamma_0+x^T\gamma))\exp(\beta_{0,k}+x^T\beta^{(k)})}{\exp(-(\gamma_0+x^T\gamma))\sum_{l=1}^K\exp(\beta_{0,k}+x^T\beta^{(l)})}\\
&=\frac{\exp(\beta_{0,k}+x^T\beta^{(k)})}{\sum_{l=1}^K\exp(\beta_{0,k}+x^T\beta^{(l)})}=P(Y=k\mid x)\\
\therefore P(Y=k\mid x)&=\frac{\exp(\beta_{0,k}+x^T\beta^{(k)}-(\gamma_0+x^T\gamma))}{\sum_{l=1}^K\exp(\beta_{0,k}+x^T\beta^{(l)}-(\gamma_0+x^T\gamma))}
\end{align*}
となり,等号が得られることがわかる.\\

\item $\beta_{j,1},\cdots,\beta_{j,K}$の値が求まったとき$(j=1,\cdots,p)$,$\displaystyle \sum_{k=1}^K|\beta_{j,k}-\gamma_j|$の値を最小にする$\gamma_j\in\mathbb{R}$を求めればよい.またそのような$\gamma_j$ は$\beta_{j,1},\cdots,\beta_{j,K}$の中央値になると文中で述べられているので,$\beta_{j,1},\cdots,\beta_{j,K}$を求めた後にそれらすべてから中央値を引けばよいことがわかる.\\

\item $\displaystyle \sum_{k=1}^K\beta_{0,k}$となるように設定されていることから,最初に$\beta_{0,1},\cdots,\beta_{0,K}$の値を求めた後,それらすべてから$\bar{\beta}:=\displaystyle \frac{1}{K}\sum_{k=1}^K\beta_{0,k}$を引いていることがわかる.\qed\\
\end{enumerate}
\end{Ex}
%%%%%%%%%%%%%%%%%%%%%%%%%%%%%%%%%%%%%%%%
\begin{Ex}
\begin{enumerate}[(a)]
\item (2.26)式は観測値$(x_1,y_1),\cdots,(x_N,y_N)$から得られる尤度関数で
$$\prod_{i=1}^N\frac{\mu_i^{y_i}}{y_i!}e^{-\mu_i}\quad (\mu_i = e^{\beta_0+x_i^T \beta})$$
となっている.この尤度関数のマイナス対数をとると
\begin{align*}
-\log\left(\prod_{i=1}^N\frac{\mu_i^{y_i}}{y_i!}e^{-\mu_i}\right)&=\sum_{i=1}^N\{\log(y_i!)-y_i\log(\mu_i)+\mu_i\}\\
&=\sum_{i=1}^N\{\log(y_i!)-y_i(\beta_0+x_i^T\beta)+e^{\beta_0+x_i^T\beta}\}\\
&=L(\beta_0,\beta)+\sum_{i=1}^N \log(y_i!)
\end{align*}
となる.最下段の$\beta_0,\beta$に関する最小化は$\frac{1}{N}L(\beta_0,\beta)$の$\beta_0,\beta$に関する最小化と同値であり,この式に正則化項をつけると
$$\frac{1}{N}L(\beta_0,\beta)+\lambda\|\beta\|_1$$
となる.このようにして(2.27)式が導出される.\\

\item $L(\beta_0,\beta)=\sum_{i=1}^N\{y_i(\beta_0+x_i^T \beta)-e^{\beta_0+ x_i^T \beta}\}$を$\beta_j(j=0,\cdots,p)$で微分すると
\begin{align*}
\frac{\partial L}{\partial \beta_j} &=\sum_{i=1}^N \{-y_i x_{i,j}+x_{i,j}e^{\beta_0+ x_i^T \beta}\}\\
&=-\sum_{i=1}^N x_{i,j}(y_i-e^{\beta_0+ x_i^T \beta})
\end{align*}
となる.ただし$x_{i,0}=1(i=1,\cdots,N)$とした.よって
\begin{align*}
\nabla L&=-\begin{pmatrix}
1 & \cdots & 1\\
x_{1,1} & \cdots & x_{N,1}\\
\vdots & \ddots & \vdots \\
x_{1,p} & \cdots & x_{N,p}
\end{pmatrix}\left(\begin{array}{c}
y_1-e^{\beta_0+x_1^T \beta}\\
\vdots\\
y_N-e^{\beta_0+x_N^T \beta}
\end{array}\right)
\end{align*}
となるので,
$$u=\left(\begin{array}{c}
y_1-e^{\beta_0+x_1^T \beta}\\
\vdots\\
y_N-e^{\beta_0+x_N^T \beta}
\end{array}\right)$$
とすれば$\nabla L = -X^T u$とかくことができる.\\

\item (b)で得られた$\frac{\partial L}{\partial \beta_j}$をさらに$\beta_k$で偏微分すると
\begin{align*}
\frac{\partial }{\partial \beta_k}\left(\frac{\partial L}{\partial \beta_j}\right)&=-\sum_{i=1}^N x_{i,j}x_{i,k}e^{\beta_0+x_i^T \beta}
\end{align*}
となることより
\begin{align*}
\nabla^2 L &=X^T \begin{pmatrix}
e^{\beta_0+x_1^T \beta} & \cdots & 0\\
\vdots & \ddots & \vdots \\
0 & \cdots & e^{\beta_0+x_N^T \beta}
\end{pmatrix}X
\end{align*}
となるので,
$$W = \begin{pmatrix}
e^{\beta_0+x_1^T \beta} & \cdots & 0\\
\vdots & \ddots & \vdots \\
0 & \cdots & e^{\beta_0+x_N^T \beta}
\end{pmatrix}$$
とすれば$\nabla^2 L =X^TWX$と書くことができる.\qed\\
\end{enumerate}
また掲載されているプログラムの空欄を埋めてコード,およびそれを実行した結果は下のようになる
\begin{lstlisting}[basicstyle = \ttfamily\footnotesize, frame = single]
def poisson_lasso(X, y, lam):
    p = X.shape[1]   # pはすべて1の列を含んでいる
    beta = np.random.randn(p)
    gamma = np.random.randn(p)
    while np.sum((beta - gamma) ** 2) > 0.0001:
        beta = gamma
        s = np.dot(X, beta)
        w = np.exp(s) #空欄1
        u = y - w #空欄2
        z = s + u / w ##空欄3
        gamma_0, gamma_1 = W_linear_lasso(X[:, range(1, p)],
                                          z, np.diag(w), lam)
        gamma = np.block([gamma_0, gamma_1]).copy()
        print(gamma)
    return gamma

N = 100    
p = 3
X = np.random.randn(N, p)
X = np.concatenate([np.ones(N).reshape(N, 1), X], axis=1)
beta = np.random.randn(p + 1)
s = np.dot(X, beta)
y = np.random.poisson(lam=np.exp(s))
print(beta)
\end{lstlisting}
\begin{lstlisting}[basicstyle = \ttfamily\footnotesize, frame = single]
[-0.47085992 -0.31067144  0.30882489 -0.00316935] #真のbの値
[ 0.7654697  0.        -0.        -0.       ]
[ 0.05384126 -0.          0.          0.        ]
[-0.35865758 -0.          0.          0.        ]
[-0.4711855 -0.         0.         0.       ]
[-0.47801239 -0.          0.          0.        ]#推定されたb
\end{lstlisting}
推定された$\beta$がスパースなものであることが確認できた.\qed\\

\end{Ex}
%%%%%%%%%%%%%%%%%%%%%%%%%

\begin{Ex}
  \begin{enumerate}[(a)]

\item 関数$L$は
$$L(\beta) := -\sum_{i:\delta_i = 1}\log \frac{e^{x_i^T \beta}}{\sum_{j\in R_i}e^{x_j^T \beta}}$$
で定義されている.この関数$L$を$\beta_k$で偏微分すると

\begin{align*}
  \frac{\partial L}{\partial \beta_k}&=-\sum_{i:\delta_i = 1}\left\{x_{i,k}-\frac{\sum_{j\in R_i x_{j,k}\exp(x_j^T \beta)}}{\sum_{h\in R_i\exp(x_h^T \beta) }}\right\}\\
  &=-\sum_{i=1}^N x_{i,k}\left\{
    \delta_i - \sum_{j\in C_i}\frac{\exp(x_i^T \beta)}{\sum_{h\in R_j}\exp(x_h^T \beta)}
  \right\}
\end{align*}
とかくことができる.ここで,$S_i\alpha= \sum_{h\in R_i}\exp(x_h^T \beta)$としたときに
\begin{align*}
\sum_{i;\delta_i=1}\sum_{j\in R_j}\frac{x_{j,k}\exp(x_j^T \beta)}{S_i}&= \sum_{i=1}^N\sum_{i\in C_i}\frac{x_{j,k}\exp(x_j^T \beta)}{S_i}\\
&=\sum_{i=1}^N x_{i,k}\sum_{j\in C_i}\frac{\exp(x_i^T \beta)}{S_j}
\end{align*}
と変形できることより
\begin{align*}
  \frac{\partial L}{\partial \beta_k}&=
  -\sum_{i=1}^N x_{i,k}\left\{\delta_i - \sum_{j\in C_i}\frac{\exp(x_i^T \beta)}{\sum_{h\in R_j}\exp(x_h^T \beta)}\right\}
\end{align*}
となることがわかる.したがって
\begin{align*}
  \nabla L&=-X^T \left(\begin{array}{c}
    \delta_1- \sum_{j\in C_1}\frac{\exp(x_1^T \beta)}{\sum_{h\in R_j}\exp(x_h^T \beta)}\\
    \vdots \\
    \delta_N- \sum_{j\in C_N}\frac{\exp(x_1^T \beta)}{\sum_{h\in R_j}\exp(x_h^T \beta)}
  \end{array}\right)
\end{align*}
となることより,求める$u$は
$$u=\left(\begin{array}{c}
  \delta_1- \sum_{j\in C_1}\frac{\exp(x_1^T \beta)}{\sum_{h\in R_j}\exp(x_h^T \beta)}\\
  \vdots \\
  \delta_N- \sum_{j\in C_N}\frac{\exp(x_1^T \beta)}{\sum_{h\in R_j}\exp(x_h^T \beta)}
\end{array}\right)$$
とかける.\\

\item $\nabla^2 L$の各成分を変形していくと以下のようになる

\begin{align*}
  \frac{\partial L}{\partial \beta_k\partial\beta_l}&=
  \sum_{i=1}^N\sum_{j\in C_i}\frac{\partial}{\partial\beta_l}\left(\frac{\exp(x_i^T\beta)}{\sum_{h\in R_j}\exp(x_h^T\beta)}\right)\\
  &=\sum_{i=1}^Nx_{i,k}\sum_{j\in C_i}\frac{1}{(\sum_{r\in R_j}\exp(x_r^T\beta))^2}\left\{x_{i,l}\exp(x_i^T\beta)\sum_{s\in R_j}\exp(x_s^T\beta)\right.\\
  &\hspace{9cm}\left.-\exp(x_i^T\beta)\sum_{h\in R_j}x_{h,l}\exp(x_h^T \beta)\right\}\\
  &=\sum_{i=1}^N\sum_{h=1}^N x_{i,k}x_{h,l}\sum_{j\in C_i}\frac{\exp(x_i^T\beta)}{(\sum_{t\in R_j}\exp(x_t^T \beta))^2}\left\{I(i=h)\sum_{s\in R_j}\exp(x_s^T \beta)-I(h\in R_j)\exp(x_h^T \beta)\right\}
\end{align*}
とかける.ここで$W$の対角成分,すなわち$i=h$となる成分は
\begin{align*}
  w_{i} &=\sum_{j\in C_i}\frac{\exp(x_i^T\beta)}{(\sum_{t\in R_j}\exp(x_t^T \beta))^2}\left\{\sum_{s\in R_j}\exp(x_s^T \beta)-I(h\in R_j)\exp(x_i^T \beta)\right\}
\end{align*}
とかくことができ,
$$\delta_i =1,j\in R_i\Leftrightarrow i\in C_i$$
であることより$i=h$ならば$j\in C_i \Leftrightarrow i=h\in R_j$が得られる.したがって
$$\pi_{i,j}:=\frac{\exp(x_i^T \beta)}{\sum_{r\in R_j}\exp(x_i^T\beta)}$$
とすることで,$W$の各対角成分$w_i$は
\begin{align*}
  w_i&=
  \sum_{j\in C_i}\frac{\exp(x_i^T \beta)}{\sum_{r\in R_j}\exp(x_i^T\beta)}\left\{1-\frac{\exp(x_i-T \beta)}{\sum_{r\in R_j}\exp(x_i^T\beta)}\right\}=\sum_{j\in C_i}\pi_{i,j}(1-\pi_{i,j})
\end{align*}
とかける.\qed\\
\end{enumerate}


\end{Ex}



\end{document}
